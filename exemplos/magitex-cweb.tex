@*0 Um Casador de Expressões Regulares.

@*1 Introdução.

Código bonito geralmente é simples---claro e fácil de entender. Código
bonito geralmente é compacto---o suficiente para fazer o trabalho bem
feito e nada mais---mas não críptico a ponto de não poder ser
entendido. Código bonito pode também ser geral, resolvendo uma ampla
classe de problemas de forma uniforme. Ele pode ser descrito como
elegante, uma amostra de bom gosto e refinamento.

Neste capítulo eu irei descrever um pedaço de código bonito, para
casar expressões regulares que se encaixa em todos estes critérios.

Expressões regulares são uma notação para descrever padrões de texto,
de fato é uma linguagem específica para casar padrões. Embora hajam
muitas variações, todas compartilham a ideia de que a maioria dos
caracteres em um padrão se casam com ocorrências literais deles
mesmos, mas alguns ``metacaracteres'' tem significados especiais, por
exemplo ``*'' indica algum tipo de repetição ou [...] significa
qualquer um dos caracteres entre os colchetes.

Na prática, a maioria das buscas em programas com editores de texto
são para palavras literais, então as expressões regulares são
geralmente textos literais como ``print'' que irá casar ``sprint'' ou
``printer'' em qualquer lugar. Nos assim chamados meta-caracteres de
nomes de arquivos em Unix e Windows, um ``*'' casa com qualquer número
de caracteres, então o padrão ``*.c'' casa com todos os arquivos que
terminam com ``.c''. Existem muitas, muitas variações das expressões
regulares, mesmo em contextos nos quais as pessoas esperariam que
fôssem as mesmas. O livro ``Dominando Expressões Regulares'' de
Jeffrey Friedl é um estudo exaustivo do tópico.

Expressões regulares foram inventadas por Stephen Kleene no meio dos
anos 50 como uma notação para autômatos finitos, e de fato elas eram
equivalentes aos autômatos finitos que elas represenavam. Expressões
regulares apareceram pela primeira vez nas configurações de um
programa na versão de ken Thompson do editor de textos QED no meio dos
anos 60. Em 1967, Ken pediu o registro de uma patente sobre um
mecanismo para encontrar rapidamene textos que casavam com uma
expressão regular; a qual foi concedida em 1971, uma das primeiras
patentes de software nos Estados Unidos.

Expressões regulares se mudaram do QED para o editor Unix ``ed'', e
então para a feramenta quintessencial do unix, o ``grep'', a qual foi
criada por Ken ao realizar uma cirurgia radical no ``ed''. Através de
todos estes programas amplamente usados, expressões regulares
tornaram-se familiares por toda a comunidade Unix inicial.

O casador original de Ken era muito rápido, pois ele combinava duas
ideias independentes. Uma era gerar instruções de máquina sob medida
durante o casamento de padrões de forma que ela executasse na
velocidade da máquina, não da interpretação. A outra era seguir
adiante com todos os padrões casados em cada estágio, assim não
haveria necessidade de voltar atrás para procurar por novos trechos de
texto que poderiam casar com a expressão. O código do casador nos
próximos editores de texto que Ken escreveu, como ``ed'', usava um
algoritmo mais simples que voltava atrás quando necessário. Na teoria
isso era mais lento mas nos padrões encontrados na prática raramente
era necessário voltar atrás, então o código de ``ed'' e ``grep'' era
bom o bastante para a maioria dos propósitos.

Os próximos casadores de expressões regulares como ``egrep'' e
'fgrep'' adicionaram classes mais ricas de expressões regulares, e se
focaram na execução rápida dos padrões, fôssem eles quais
fôssem. Expressões regulares ainda mais complexas tornaram-se
populares, e eram incluídas não só em biblioecas baseadas em C, mas
também como parte da sintaxe de linguagens de script como Awk e Perl.

@*1 A Prática de Programação.

Em 1998, Rob Pike e eu estávamos escrevendo ``A Prática da
Programação''. O último capítuulo do livro, ``Noação'', era uma
coleção de um número de exemplos nos quais uma boa noação levou a
melhores programas e melhores programações. Isso incluia o uso de
estruturas de dados simples (como o formato do ``printf'') e a geração
de código à partir de tabelas.

Dada a nossa experiência em Unix e os muitos anos de experiência com
ferramentas baseadas na notação de expressões regulares, nós
naturalmente queríamos incluir uma discussão de expressões regulares,
e parecia obrigatório que incluíssemos uma implementação também. Dada
a nossa ênfase em ferramentas, também parecia melhor focar na classe
de expressões regulares encontradas no ``grep'' ao invés dos
caracteres especiais do shell,assim nós também poderíamos falar sobre
o próprio projeto do ``grep''.

O problema era que qualquer pacote existente de casar expressões
regulares era grande demais. O ``grep'' local tinha 500 linhas (cerca
de 10 páginas do livro). Pacotes de expressão regular em software
livre tendiam a ser enormes, basicamente tinham o tamanho do livro
inteiro, pois eles haviam sido projetados para a generalidade,
flexibilidade e velocidade; nenhum era adequado para a pedagogia.

Eu sugeri a Rob que nós precisávamos encontrar o menor pacote de
expressões regulares que ilustrassem as ideias básicas enquanto ainda
reconhecêssem uma classe não-trivial e útil de padrões. Idealmente o
código deveria caber em uma única página.

Rob desapareceu em seu escritório, e pelo menos como eu me lembro,
apareceu novamente em não mais do que uma ou duas horas com as 30
linhas de código C que apareceriam no Capítulo 9 do livro. Aquele
código implementa um marcador de expressão regular que suporta os
seguintes elementos:

\halign{\hfil # & \espaco{1cm} # \hfil \cr
c & Casa com qualquer caractere literal c. \cr
. & Casa com um único caractere qualquer. \cr
\^ & Casa com o começo de uma cadeia de caracteres. \cr
\$ & Casa com o fim de uma cadeia de caracteres.\cr
* & Casa com zero ou mais ocorrências do caractere anterior.\cr}

Esa é uma classe de casos bastante útil; em minha própria experiência
de uso de expressões regulares, isso facilmente abrange 95\% de todos
os casos. Em muitas situações, resolver o problema certo é um grande
passo no caminho para um programa bonito. Rob merece crédito por
escolher tão sabiamente dentre tantas opções, um conjunto de
funcionalidade muito pequena, mas bem-definida e extensível.

A implementação de Rob em si é um exemplo perfeito de código bonito:
compacto, elegante, eficiente e útil. É um dos melhores exemplos de
recursão que eu já vi e mostra o poder dos ponteiros em C. Embora na
época nós estivéssemos mais interessados em mostrar o papel importante
de uma boa notação em tornar um programa mais fácil de usar e talvez
mais fácil de escrever também, o código de expressão regular também
foi uma forma excelente de ilustrar algoritmos, estruturas de dados,
melhoria de performance e ouros tópicos importantes.

@*1 Implementação.

No livro, o casador de expressões regulares é parte de um programa que
imita ``grep'', mas o código de expressão regular é completamente
separável de sua vizinhança. O programa principal não é interessante
aqui---ele simplesmente lê da entrada padrão ou de uma sequencia de
arquivos, e imprime aqueles cujas linhas contém texto que casa com a
expressão regular, assim como o ``grep'' originaç e muitas outras
ferramentas Unix.

Este é o código de casamento:

@<Casador de Expressões Regulares@>=
/* match: search for regexp anywhere in text */
int match(char *regexp, char *text){
  if(regexp[0] == '^')
    return matchhere(regexp+1, text);
  do{    /* must look even if string is empty */
    if(matchhere(regexp, text))
      return 1;
  } while (*text++ != '\0');
  return 0;
}

/* matchhere: search for regexp at beginning of text */
int matchhere(char *regexp, char *text){
  if (regexp[0] == '\0')
    return 1;
  if (regexp[1] == '*')
    return matchstar(regexp[0], regexp+2, text);
  if (regexp[0] == '$' && regexp[1] == '\0')
    return *text == '\0';
  if (*text!='\0' && (regexp[0]=='.' || regexp[0]==*text))
    return matchhere(regexp+1, text+1);
  return 0;
}

/* matchstar: search for c*regexp at beginning of text */
int matchstar(int c, char *regexp, char *text){
  do {    /* a * matches zero or more instances */
    if (matchhere(regexp, text))
      return 1;
  } while (*text != '\0' && (*text++ == c || c == '.'));
  return 0;
}
@

A função |match(regexp, text)| testa se há uma ocorrência
da expressão regular em qualquer lugar dentro do texto; ela retorna 1
se um casamento é encontrado e 0 caso contrário. Se há mais de um
casamento, ela encontra o mais à esquerda e menor.

A operação básica de |match| é direta. Se o primeiro
caractere da expressão regular é \^, qualquer casamento possível deve
ocorrer no começo da cadeia de caracteres. Isto é, se a expressão
regular é \^xyz, ela deve casar ``xyz'' só se ``xyz'' ocorrer no
começo do texto, não no meio. Isso é testafo casando o resto da
expressão regular somente com o começo do texto, e não com o resto.

Fora deste caso, a expresão regular iria casar com qualquer padrão
dentro da string. Isso é testado casando o padrão contra cada posição
de caractere do texto. Se há múltiplos casamentos, somente o primeiro
será identificado.

Note que avançar na cadeia de caracteres de entrada é feito com um
laço |do-while|, algo relativamente incomum na construção
de programas em C.A ocorrência de um |do-while| deve sempre
levantar uma questão: porque a condição de terminar o laço não é
testada no começo do laço, antes que seja tarde demais? Mas o teste
está correto aqui: como o operador * permite casar cadeias de
caractere vazias, nós primeiro temos que testar se um casamento vazio
é possível.

O grosso do trabalho é feito na função |matchhere(regexp, text)|, a
 qual testa se a expressão regular casa com o texto que começa logo
 ali. A função opera tentando casar o primeiro caractere da expressão
 regular com o primeiro caractere de texto. Se o casamento falhar, não
 pode haver casamento nesta posição do texto e |matchhere| retorna
 0. Se o casamento for bem-sucedido, contudo, é possível avançar para
 o próximo caractere da expressão regular e o próximo caractere do
 texto. Isso é feito chamando |matchhere| recursivamente.

A situação é um pouco mais complicada por causa de alguns casos
especiais, e é claro, a necessidade de interromper a recursão. O caso
mais fácil é quando a expressão regular está no seu fim (|regexp[0] ==
'\\ 0'|), então todos os testes anteriores foram bem-sucedidos, e
assim a expressão regular casa com o texto.

Se uma expressão regular é um caracere seguido por *, |matchstar| é
chamada para ver se a palavra casa com o fechamento. A
função |matchstar(c, regexp, text)| tenta casar repetições
do caractere ``c'', começando com zero repetições e contando atéque
consiga casar com o restante do texto ou falhe e conclua não ser
possível um casamento. Isso identifica um ``menor casamento'', o que é
bom para simples casamentos de padrões como em ``grep'', onde tudo o
que importa é encontrar o casamento o mais rápido possível. Um ``maior
casamento'' é mais intuitivo e quase certamente melhor para umeditor
de textos onde o texto casado será substituído. A maioria das
bibliotecas de expressões regulares fornece ambas as alternativas, e o
nosso exemplo apresenta uma simples variante de |matchstar|
para este caso, mostrada depois.

Se a expressão regular é um \$ no im da expressão, então o texto casa
somente se estiver no fim.

Caso contrário, se nós não esamos no fim da cadeia de caracteres e se
o primeiro caractere do texto casa com o primeiro caractere da
expressão regular, tudo continua seguindo bem e iremos então testar se
o próximo caractere de texto também casa com o próximo caractere da
expressão regular fazendo uma chamada recursiva para |matchhere|. Esta
chamada recursiva é o coração do algoritmo e o por quê do código ser
tão compacto e limpo.

Se todas estas tentativas falharem, então não pode haver um casamento
neste ponto na expressão regular, e então |matchhere| retorna 0.

Este código definitivamente usa ponteiros de C. Em cada estágio da
recursão, se alguma coisa casa, a chamada recursiva que se segue usar
aritmética de ponteiros para que a próxima função seja chamada com o
próximo caractere da expressão regular e próximo caractere do texto. A
profundidade da recursão não será maior que o tamanho do padrão, o que
é normal se o seu uso for pequeno, mas que deixa o risco para
esgotarmos todo o espaço da pilha.

@*1 Alternativas.

Esta é uma porção de código muito elegante e bem-escrita, mas não é
perfeita. O que pode ser feito diferente? Pode-se reorganizar
|matchhere| para lidar primeiro com \$ antes de *. Embora não faça
diferença aqui, me parece mais natural, e uma boa regra é cuidar dos
casos simples antes dos mais difíceis.

Mas via de regra, contudo, a ordem dos testes é crítica. Por exemplo,
neste teste de |matchstar|:

@<Trecho de matchstar@>=
    } while (*text != '\0' && (*text++ == c || c == '.'));
@

não importa o que aconteça, nós devemos sempre avançar para o próximo
caractere da cadeia de caracteres, então o incremento em
|text++| deve sempre ser feito.

Este código é cuidadoso com as condições de término. Geralmente, o
sucesso de um casamento é determinado quando terminamos de percorrer a
expressão regular ao mesmo tempo que o texto. Se ambos acabam ao mesmo
tempo, isso indica um casamento; se um termina antes do outro, não há
casamento. Isso é talvez mais óbvio em uma linha como:

@<Condição de Parada@>=
  if (regexp[0] == '$' && regexp[1] == '\0')
    return *text == '\0';
@

mas condições de parada mais sutis aparecem em outros lugares também.

Esta versão de |matchstar| que implementa o maior casamento mais à
esquerda começa identificando uma sequencia máxima de ocorrências do
caractere de entrada ``c''. Então, ela usa |matchhere| para tentar
extender o casamento para o resto dos padrões e o resto do texto. Cada
falha reduz o número de ``c''s em um e tenta-se novamente, incluindo o
caso de zero ocorrências.

@<Implementação alternativa de matchhere@>=
/* matchstar: leftmost longest search for c*regexp */
int matchstar(int c, char *regexp, char *text)
{
  char *t;

  for (t = text; *t != '\0' && (*t == c || c == '.'); t++)
  ;
  do { /* * matches zero or more */
    if (matchhere(regexp, t))
    return 1;
    } while (t-- > text);
  return 0;
}
@

Considere a expressão regular ``(.*)'', que casa qualquer texto
arbitrário entre parênteses. Dado o texto alvo

\alinhaverbatim
for (t = text; *t != '\\ 0' && (*t == c || c == '.'); t++)
\alinhanormal

um casamento maior logo no começo irá identificar a expressão inteira
entre parênteses, enquanto um casamento menor iria parar ao encontrar
o primeiro fechamento de parênteses. (É claro, um casamento maior se
começar da segunda abertura de parênteses também iria se extender até
o fim do texto).

@*1 Estendendo o Exemplo.

O propósito de nosso livro era ensinar boas práticas de
programação. Na época em que o livro foi escrito,Rob e eu estávamos no
Bell Labs, assim não tínhamos experiência em primeira mão de como o
livro seria usado em uma sala de aula. Foi gratificante descobrir que
alguns dos materiais se sairam bem usados em aulas. Eu venho usado
este código deste 2000 como um meio de ensinar importantes detalhes
sobre programação.

Em primeiro lugar, ele mostra como recursão é útil e leva a código
limpo de forma nova. Não é mais um exemplo de Quicksort (ou
fatorial!), nem é algum tipo de código para percorrer uma árvore.

É também um bom exemplo de experimento com performance. Sua
performance não é muito diferente das versões de sistema encontradas
no ``grep'', o que mostra que a técnica recursiva não é muito cara e
que não vale à pena tentar removê-la para deixar o código mais rápido.

Por outro lado, é também uma ótima demonstração da importância de um
bom algoritmo. Se um padrão incluir muitos ``.*''s, a implementação
mais direa precisa ficar voltando pra trás muito e em alguns casos
realmente irá executar lentamente. (O ``grep'' padrão do Unix tem a
mesma propriedade). Por exemplo, o comando

\alinhaverbatim
grep 'a.*a.*a.*a.a' 
\alinhanormal

leva cerca de 20 segundos para processar um arquivo de texto com 4MB
em uma máquina típica. Uma implementação baseada em converter um
autômato finito não-deerminístico para um automato determinístico,
como em ``egrep'', terá uma performance muito melhor em casos
difíceis---o mesmo padrão e a mesma entrada é processada em menos de
um décimo de segundo., e no geral, o tempo de execução é independente
do padrão.

Exensões para esa classe de expressões regulares pode formar o ponto
de partida de uma série de exercícios. Por exemplo:

(1) Adicione outros metacaracteres, como ``+'' para uma ou mais
ocorrências do caractere anterior, ou ``?'' para zero ou uma
ocorrência. E alguma forma de escapar os metacaracteres para que eles
possam ser usados como representação literal.

(2) Separe o processamento da expressão regular em uma fase de
``compilação'' e uma fase de ``execução''. A compilação converte a
expressão regular em uma forma interna que faz com que o código de
casamento seja mais simples e assim futuros casamentos executem mais
rápido. Esta separação não é necessária para a classe simples de
expressões regulares do projeto original, mas faz sentido em
aplicações como ``grep'' onde a classe é mais rica e a mesma expressão
regular é usada em um grande número de linhas de entrada.

(3) Adicione classes de caractere como [abc] e [0-9], as quais na
notação convencional de ``grep'' casam ``a'' ou ``b'' ou ``c'' e um
dígito respectivamente. Isso pode ser feito de muitas formas, a mais
natural seria substituir os |char *|s do código original
por uma estrutura:

@<Estrutura de Token de Expressão Regular@>=
typedef struct RE {
  int     type;   /* CHAR, STAR, etc. */
  char    ch;     /* the character itself */
  char    *ccl;   /* for [...] instead */
  int     nccl;   /* true if class is negated [^...] */
} RE;
@

e modificar o código básico para lidar com uma cadeia destas
estruturas ao invés de uma cadeia de caracteres. Não é estritamente
necessário separar a compilação da execução para esta situação, mas
fazendo isso, a tarefa fica mais fácil. Estudantes que seguem o
conselho de pré-compilar em tais estruturas invariavelmente se saem
melhor do que aqueles que tentam interpretar algum padrão complicado
de estrutura de dados na hora da execução.

Escrever especificações claras e sem ambiguidades para classes de
caracteres é difícil, e implementá-las perfeitamente é pior,
requerendo um monte de código edioso e pouco instrutivo. Eu
simplifiquei este exercício ao longo do tempo, e hoje mais
frequentemente peço para as abreviações de Perl como \\\/d para dígito
e \\\/D para não-dígito ao invés da notação de intervalos entre
colchetes.

(4) Use algum tipo opaco para esconder a estrutura |RE| e todos os
detalhes de implementação. Esta é uma boa forma de mostrar programação
orientada à objetos em C, que não suporta muito mais além dissos. De
fato, na prática usa-se uma classe para expressões regulares mas com
nomes de funções tais como |RE_new()| e |RE_match()| para os métodos
ao invés do açúcar sintático de linguagens orientadas à objeto.

(5) Modifique a classe de expressões regulares para funcionarem como
os asteriscos de vários shells: os casamentos precisam ocorrer em todo
o exto e ``*'' casa com qualquer número de caracteres enquanto ``?''
casa com um único caractere. Pode-se modificar o algoritmo ou mapear a
entrada no algoritmo existente.

(6) Converta o código para Java. O código original usa ponteiros de C
muito bem, mas é uma boa prática encontrar alternativas em uma
linguagem diferente. As versões em java usam ou |String.charAt|
(indexando ao invés de usar ponteiros) ou |String.substring| (mais
próxima da versão com ponteiros). Nenhuma delas é tão clara como o
código em C, e nenhuma é tão compacta. Embora performance não seja
mesmo parte do exercício, é interessante notar que a implementação em
Java executacerca de seis ou sete vezes mais devagar que as versões em
C.

(7) Escreva uma classe que encapsule tudo e converta internamente de
expressões regulares para as classes de casamento de padrão do Java,
as quais separam a compilação e o casamento de forma bem
diferente. Este é um bom exemplo do padrão Adaptador, que dá um rosto
diferente para uma classe ou conjunto de funções existente.

Eu também usei este código extensivamente para explorar técnicas de
teste de código. Expressões regulares são ricas o bastante para que
testá-las não seja trivial, mas pequenas o bastante para que uma
pessoa possa rapidamente escrever uma coleção substancial de testes a
serem feitos automaticamente. Para as extensões listadas acima, eu
peço para que estudantes escrevam um grande número de testes em uma
linguagem compacta (outro exemplo de ``notação'') e usem estes testes
em seu próprio código; naturalmente eu uso os testes deles nos códigos
de outros estudantes também.

@*1 Conclusões.

Eu fui surpreendido em quão compacto e elegante este código era quando
Rob Pike escreveu ele pela primeira vez---era muito menor e mais
poderoso que eu achei ser possível. Pensando sobre isso, pode-se ver
um número de razões so por quê este código é tão pequeno. 

Primeiro, as funcionalidades foram bem escolhidas para serem as mais
úteis e darem o maior destaque na implementação, sem qualquer
obstáculo. Por exemplo, a implementação dos padrões ``\^'' e ``\$''
precisou de apenas 3 ou 4 linhas, mas mostram como lidar com casos
especiais de maneira limpa antes de lidar com casos gerais de maneira
uniforme. A operação de fechamento * é uma noção fundamental de
expressões regulares e fornece a única forma de lidar com padrões de
tamanhos desconhecidos, então tinha que estar presente. Mas adicionar
coisas como ``+'' e ``?'' não acrescentaria nenhum grande ensinamento,
então estas coisas são deixadas como exercícios.

Segundo, recursão vale à pena. Esta técnica de programação fundamental
quase sempre leva á código menor, mais limpo e mais elegante que o
escrito por meio de laços explícitos e este foi o caso aqui. A ideia
de ir removendo um caractere do texto e um da expressão e ir
percorrendo os demais recursivamente imita a estrutura recursiva do
tradicional fatorial ou do tamanho de string, mas em uma maneira muito
mais interessante e útil.

Rob me disse que a recursão não era tanto uma decisão explícita de
projeto, mas uma consequência de como ele lidou com o problema: dado
um padrão e um texto, escreva uma função que procure um casamento; o
que por sua vez gerou uma função |matchhere|; e assim por
diante.

``Eu tenho memórias vívidas de assistir o código quase se escrever
assim sozinho. O único desafio era obte as condições limite adequadas
para se sair da recursão. Colocado de outro modo, a recursão não é
apenas o método de implementação, é também um reflexo do processo de
pensamento que tomei ao escrever o código, o qual foi em parte
responsável pela simplicidade do código. o mais importante, talvez, é
que eu não tinha um projeto quando comecei, eu apenas iniciei a
programar e vi o que se desenvolveu. De repente, estava feito.''

Terceiro, este código realmente usa as estruturas da linguagem
efetivamente. Ponteiros podem ser abusados, é claro, mas aqui eles são
usados para criar expressões compactas que naturalmente expressam a
extração de caracteres individuais e o avanço para o próximo
caractere. O mesmo efeito pode ser obtido usando os índices de vetores
e substrings, mas neste código ponteiros fazem um trabalho melhor,
especialmente quando unidos às estruturas da linguagem X para
auto-incremento e conversão implícita de valores verdade.

Eu não conheço qualquer outro pedaçõ de código que faz tanto em tão
poucas linhas ao mesmo tempo em que fornece uma rica fonte de
ensinamento e ideias.

\fim
