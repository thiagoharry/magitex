\alinhaversocentro\negrito{Manifesto do Partido Comunista}
\alinhanormal

\espaco{0,5cm}

Um espectro ronda a Europa---o espectro do comunismo. Todas as
potências da velha Europa unem-se numa Santa Aliança para conjurá-lo:
o papa e o Tzar, Metternich e Guizot, os radicais da França e os
policiais da Alemanha.

Que partido de oposição não foi acusado de comunista por seus
adversários no poder? Que partido de oposição, por sua vez, não lançou
a seus adversários de direita ou de esquerda a pecha infamante de
comunista?

Duas conclusões decorrem desses fatos:

1°) O comunismo já é reconhecido como força por todas as potências da
Europa;

2°) É tempo de os comunistas exporem, à face do mundo inteiro, seu
modo de ver, seus fins e suas tendências, opondo um manifesto do
próprio partido à lenda\rodape{Na edição de 1848: lendas.} do
espectro do comunismo.

Com este fim, reuniram-se, em Londres, comunistas de várias
nacionalidades e redigiram o manifesto seguinte, que será publicado em
Inglês, francês, alemão, italiano, flamengo e dinamarquês.

\espaco{0,25cm}

\negrito{Burgueses e Proletários}

\espaco{0,25cm}

A história de todas as sociedades que existiram até aos nossos
dias\rodape{Ou mais exatamente a história escrita. Em 1847, a história
da organização social que precedeu toda a história escrita, a
pré-história, era quase desconhecida. Posteriormente, Haxthausen
descobriu na Rússia a propriedade comum da terra. Maurer demonstrou
que esta foi a base social da qual partiram historicamente todas as
tribos alemãs, e foi-se descobrindo pouco a pouco que a comunidade
rural, com posse coletiva da terra, foi a forma primitiva da
sociedade, desde as Índias até à Irlanda. Finalmente, a estrutura
desta sociedade comunista primitiva foi posta a claro, no que tem de
típico, com a descoberta de Morgan que fez conhecer a verdadeira
natureza da gens e o seu lugar na tribo. Com a dissolução destas
comunidades primitivas começa a divisão da sociedade em classes
distintas, e finalmente antagônicas.} é a história da luta de classes.

Homens livres e escravos, patrícios e plebeus, senhores e servos,
mestres e oficiais, numa palavra: opressores e oprimidos, em oposição
constante, travaram uma guerra ininterrupta, ora aberta, ora
dissimulada, uma guerra que acaba sempre pela transformação
revolucionária de toda a sociedade, ou pela destruição das duas
classes beligerantes.

Nas primeiras épocas históricas, constatamos, quase por toda a parte,
uma organização completa da sociedade em classes distintas, uma escala
gradual de condições sociais: na Roma antiga, encontramos patrícios,
cavaleiros plebeus e escravos; na Idade Média, senhores feudais,
vassalos, mestres, oficiais e servos, e, além disso, em quase todas
estas classes encontramos graduações especiais.

A sociedade burguesa moderna, que saiu das ruínas da sociedade feudal,
não aboliu os antagonismos de classes. Apenas substituiu as velhas
classes, as velhas condições de opressão, as velhas formas de luta por
outras novas.

Entretanto, o caráter distintivo da nossa época, da época da
burguesia, é o de ter simplificado os antagonismos de classes. A
sociedade divide-se cada vez mais em dois vastos campos inimigos, em
duas grandes classes diametralmente opostas: a burguesia e o
proletariado.

Dos servos da Idade Média nasceram os vilãos livres das primeiras
cidades; deste estrato urbano saíram os primeiros elementos da
burguesia.

A descoberta da América e a circum-navegação da África ofereceram à
burguesia em ascensão um novo campo de atividade. Os mercados das
Índias Orientais e da China, a colonização da América, o comércio
colonial, a multiplicação dos meios de troca e das mercadorias em
geral imprimiram ao comércio, à navegação e à indústria um impulso até
então desconhecido e aceleraram com isso o desenvolvimento do elemento
revolucionário da sociedade em decomposição.

O antigo modo de exploração feudal ou cooperativo da indústria já não
podia satisfazer a procura, que crescia com a abertura de novos
mercados. A manufatura tomou o seu lugar. a média burguesia industrial
suplantou os mestres das corporações; a divisão do trabalho entre as
diferentes corporações deu lugar à divisão do trabalho no seio da
mesma oficina.

Mas os mercados cresciam sem cessar: a procura crescia sempre. A
própria manufatura torna-se insuficiente. O vapor e a máquina
revolucionaram então a produção industrial. A grande indústria moderna
suplantou a manufatura: a média burguesia deu lugar aos milionários da
indústria, aos chefes de verdadeiros exércitos industriais, aos
burgueses modernos.

A grande indústria criou o mercado mundial, preparado pela descoberta
da América. O mercado mundial acelerou prodigiosamente o
desenvolvimento da navegação e de todos os meios de transporte
terrestre. Este desenvolvimento influiu por sua vez na extensão da
indústria; e à medida que a indústria, o comércio, a navegação e os
caminhos de ferro se desenvolviam, a burguesia crescia, decuplicando
os seus capitais e reelegendo para segundo plano todas as classes
ligadas pela Idade Média.

A burguesia moderna, como vimos, é ela mesma o produto de um longo
desenvolvimento, de uma série de revoluções no modo de produção e
troca.

Cada etapa da evolução percorrida pela burguesia era acompanhada pelo
correspondente progresso político. Estrato oprimido pelo despotismo
feudal; associação armada e autônoma na comuna, uns sítios, republica
urbana independente, noutros, terceiro estado tributário da
monarquia; depois, durante o período da manufatura, contrapeso da
nobreza nas monarquias feudais ou absolutas e, em geral, pedra angular
das grandes monarquias, a burguesia, depois do estabelecimento da
grande indústria e do mercado mundial, conquistou finalmente a
hegemonia exclusiva do poder político no estado representativo
moderno. O governo do estado moderno não é mais do que uma junta que
administra os negócios comuns de toda a classe burguesa.

A burguesia desempenhou na história um papel eminentemente
revolucionário.

Onde quer que conquistou o poder, a burguesia destruiu todas as
relações feudais, patrimoniais e idílicas. Todos os laços complexos e
variados que unem o homem feudal aos seus ``superiores naturais'',
esmagou-os sem piedade para não deixar subsistir outro vínculo entre
os homens que o frio interesse, as duras exigências do ``a
contado''. Afagou o sagrado êxtase do fervor religioso, o entusiasmo
cavalheiresco e o sentimentalismo pequeno-burguês nas águas geladas do
calculo egoísta. Fez da dignidade pessoal um simples valor de troca;
substituiu as liberdades tão afetuosamente conquistadas por uma
liberdade única e impiedosa: a liberdade do comércio. Numa palavra, em
lugar da exploração velada por ilusões religiosas e políticas,
estabeleceu uma exploração, descarada, direta e brutal.

A burguesia despojou da sua auréola todas as atividades que até ai
passavam por veneráveis e dignas de piedoso respeito. Converteu o
médico, o jurista, o padre, o poeta, o sábio em assalariados ao seu
serviço.

A burguesia rasgou o véu de emocionante sentimentalismo que cobria as
relações familiares e reduziu-as a simples relações de dinheiro.

A burguesia revelou como a brutal manifestação de forças na Idade
Média, tão admirada pela reação, tinha o seu complemento natural na
preguiça mais sórdida. Foi ela que, pela primeira vez, demonstrou o
que pode realizar a atividade humana; criou maravilhas que ultrapassam
de longe as pirâmides do Egito, os aquedutos romanos, as catedrais
góticas, realizou expedições que deixaram na sombra as invasões e as
cruzadas.

A burguesia não pode existir sem revolucionar constantemente os
instrumentos de produção, e, por conseguinte, as relações de produção,
isto é, o conjunto das relações sociais. A conservação do antigo modo
de produção era, pelo contrário, a primeira condição de existência de
todas as classes industriais anteriores. Um revolução continua na
produção, uma incessante comoção de todo o sistema social, uma
agitação e uma insegurança constantes distinguem a época burguesa de
todas as anteriores. Todas as relações sociais estancadas e
ferrugentas, com o seu cortejo de concepções e de idéias antigas e
veneradas, dissolvem-se; as que as substituem envelhecem antes de se
terem podido ossificar. Tudo o que tinha solidez e permanência
esfumam-se; tudo o que era sagrado é profano, e os homens, finalmente,
vêem-se forçados a encarar as suas condições de existência e as suas
relações recíprocas com olhos desiludidos.

Impelida pela necessidade de dar cada vez maior saída aos seus
produtos, a burguesia invade o mundo inteiro. Necessita implantar-se
por toda a parte, explorar por toda a parte, estabelecer relações por
toda a parte.

Pela exploração do mercado mundial, a burguesia deu um caráter
cosmopolita à produção e ao consumo de todos os países. Para grande
desespero dos reacionários, retirou à industria a sua base
nacional. As velhas industrias nacionais foram e estão continuamente a
ser destruídas. São suplantadas por novas indústrias, cuja adoção se
torna uma questão de vida ou de morte para todas as nações
civilizadas, indústrias que já não empregam matérias-primas indígenas,
mas matérias-primas vinda das mais longínquas regiões do mundo, e
cujos produtos se consomem não só no próprio país, mas em todas as
partes do globo. Em vez das antigas necessidades, satisfeitas com
produtos nacionais, surgem necessidades novas, que reclamam para sua
satisfação produtos das regiões e climas mais longínquos. Em vez do
antigo isolamento das regiões e nações que se bastavam a si mesmas,
estabelece-se um intercâmbio universal, uma interdependência universal
das nações. E isto refere-se tanto à produção material, como à
produção intelectual. A produção intelectual de uma nação converte-se
em propriedade comum de todas. A estreiteza e o exclusivismo nacionais
tornam-se de dia para dia mais impossíveis; e da multiplicidade das
literaturas nacionais e locais nasce uma literatura universal.

Em virtude do rápido aperfeiçoamento dos instrumentos de produção e do
constante progresso dos meios de comunicação, a burguesia arrasta na
corrente da civilização todas as nações, até as mais bárbaras. Os
baixos preços das suas mercadorias constituem a artilharia pesada que
derruba todas as muralhas da China e faz capitular os bárbaros mais
fanaticamente hostis aos estrangeiros. Sob pena de corte, força todas
as nações a adotar o modo burguês de produção; força-as a introduzir a
chamada civilização, quer dizer, a tornar-se burguesas. Numa palavra:
forja um mundo à sua imagem e semelhança.

A burguesia submeteu o campo ao domínio da cidade. Criou cidades
enormes; aumentou prodigiosamente a população das cidades em
comparação com a do campo, subtraindo uma grande parte da população ao
embrutecimento da vida rural. Do mesmo modo que submeteu o campo à
cidade, os países bárbaros e semi-bárbaros aos países civilizados,
submeteu os povos de camponeses aos povos de burgueses, o Oriente ao
Ocidente.

A burguesia suprime cada vez mais o fracionamento dos meios de
produção, da propriedade e da população. Aglomerou a população,
centralizou os meios de produção e concentrou a propriedade num
pequeno número de mãos. Províncias independentes, ligadas entre si
quase unicamente por laços federais, com interesses, leis, governos e
tarifas aduaneiras diferentes, foram reunidas numa só nação, com um só
governo, uma só lei, um só interesse nacional de classe e uma só linha
alfandegária.

A burguesia, com a sua dominação de classe, que conta apenas com um
século existência, criou forças produtivas mais abundantes e mais
grandiosas que todas as gerações passadas tomadas em conjunto. A
domesticação das forças da natureza, as máquinas, a aplicação da
química à indústria e à agricultura, a navegação a vapor, os caminhos
de ferro, os telégrafos elétricos, o arroteamento de continentes
inteiros, a regularização dos rios, populações inteiras brotando da
terra - qual dos séculos passados pôde sequer suspeitar que
semelhantes forças produtivas dormitassem no seio do trabalho social?

Vimos, pois, que os meios de produção e de troca, sobre cuja base se
formou a burguesia, foram criados no interior da sociedade feudal. Ao
alcançar um certo grau de desenvolvimento, estes meios de produção e
de troca, as condições em que a sociedade feudal produzia e trocava,
toda a organização feudal da agricultura e da indústria manufaturaria,
numa palavra, as relações feudais de propriedade, deixaram de
corresponder às forças produtivas em pleno desenvolvimento. Travavam a
produção em vez de a fazer progredir, transformaram-se em outras
tantas cadeias. Era preciso quebrar essas cadeias e elas foram
quebradas.

Em seu lugar estabeleceu-se a livre concorrência, com uma constituição
social e política apropriada, com a supremacia econômica e política da
burguesia.

Hoje, produz-se diante dos nossos olhos um movimento análogo. As
relações burguesas de produção e de troca, as relações burguesas de
propriedade, toda esta sociedade burguesa moderna, que fez surgir tão
poderosos meios de produção e de troca, assemelha-se ao mago que já
não é capaz de dominar as potências infernais que desencadeou. Desde
há dezenas de anos, a história da indústria e do comércio não é mais
do que a história das forças produtivas modernas contra as atuais
relações de produção, contra as relações de produção que condicionam a
existência da burguesia e a sua dominação. Basta mencionar as crises
comerciais que, com o seu retorno periódico ameaçam, cada vez mais, a
existência de toda a sociedade burguesa. Cada crise destrói
regularmente não só uma parte considerável dos produtos já criados,
mas ainda uma grande parte das próprias forças produtivas já
existentes. Durante as crises, abate-se sobre a sociedade uma epidemia
que, em qualquer época anterior pareceria absurda - a epidemia da
superprodução. A sociedade encontra-se subitamente retrotraída a um
estado de barbárie momentânea: dir-se-ia que a fome, que uma guerra
devastadora mundial a privaram de todos os meios de subsistência; a
indústria e o comércio parecem aniquilados. E tudo isto porquê? Porque
a sociedade possui demasiada civilização, demasiados meios de vida,
demasiada industria, demasiado comércio. As forças produtivas de que
dispõe não servem já o desenvolvimento da civilização burguesa e das
relações de produção burguesas; pelo contrário, tornaram-se demasiado
poderosas para estas relações, que constituem um obstáculo ao seu
desenvolvimento; e todas as vezes que as forças produtivas sociais
vencem este obstáculo, precipitam na desordem toda a sociedade
burguesa e ameaçam a existência da propriedade burguesa. As relações
burguesas tornaram-se demasiado estreitas para conter as riquezas
criadas no seu seio. Como é que a burguesia vence estas crises? Por um
lado, destruindo pela violência uma grande quantidade de forças
produtivas, por outro lado, pela conquista de novos mercados e pela
exploração mais intensa dos antigos. A que conduz isto? A preparar
crises mais gerais e mais violentas e a diminuir os meios de
preveni-las.

As armas de que a burguesia se serviu para derrubar o feudalismo
voltaram-se agora contra a própria burguesia.

Mas a burguesia não forjou apenas as armas que a levarão à morte;
produziu também os homens que empunharão essas armas: Os operários
modernos, os proletários.

À medida que cresce a burguesia, quer dizer, o Capital, desenvolve-se
também o proletariado, a classe dos operários modernos, que não vivem
senão na condição de encontrarem trabalho e que só o encontram se o
seu trabalho aumentar o capital. Estes operários, obrigados a
vender-se dia a dia, são uma mercadoria, um artigo de comércio como
qualquer outro, sujeito, portanto, a todas as vicissitudes da
concorrência, a todas as flutuações do mercado.

O emprego crescente das máquinas e a divisão do trabalho, fazendo
perder ao trabalho do proletário todo o caráter de autonomia, fizeram,
consequentemente, que ele perdesse todo o atrativo para o
operário. Este converte-se num simples apêndice da máquina e só se lhe
exige as remunerações mais simples, mais monótonas e de mais fácil
aprendizagem. Portanto, o que custa o operário reduz-se pouco mais ou
menos ao custo dos meios de subsistência indispensáveis para viver e
perpetuar a sua descendência. Mas o preço da força de trabalho, como o
de toda a mercadoria, é igual ao seu custo de produção. Por
conseguinte quanto mais fastidioso é o trabalho, mais baixos são os
salários. Mais ainda, quanto mais se desenvolvem a maquinaria e a
divisão do trabalho, mais aumenta a quantidade de trabalho, quer
mediante o prolongamento da jornada de trabalho, quer pelo aumento do
trabalho exigido num tempo determinado, pela aceleração das cadências
das máquinas, etc.

A indústria moderna transformou a pequena oficina do mestre-artesão
patriarcal na grande fábrica do capitalista industrial. Massas de
operários, comprimidos na fábrica, estão organizados de forma
militar. Soldados rasos da industria, estão colocados sob a vigilância
de uma hierarquia completa de oficiais e sargentos. Eles não são
apenas os escravos da classe burguesa, do Estado burguês, como ainda
diariamente, a todas as horas, os escravos da máquina, do
contramestre, e sobretudo do próprio burguês fabricante. E este
despotismo é tanto mais mesquinho, odioso e exasperante, quanto maior
é a fraqueza com que proclama que tem como único fim o lucro.

Quanto menos habilidade e força requer o trabalho manual, quer dizer,
quanto maior é o desenvolvimento da industria moderna, maior é a
produção em que o trabalho dos homens é suplantado pelo das mulheres e
crianças. No que respeita à classe operária, as diferenças de idade e
sexo perdem toda a significação social. Não há senão instrumentos de
trabalho, cujo custo varia segundo a idade e o sexo.

Uma vez que o operário sofreu a exploração do fabricante e que lhe foi
pago o seu salário, converte-se em vitima doutros membros da
burguesia: o proprietário, o retalhista, o prestamista, etc.

Pequenos industriais, pequenos comerciantes e rendeiros, artesãos e
camponeses, todo o escalão inferior das classes médias de outrora,
caem nas fileiras do proletariado; uns porque os seus pequenos
capitais não lhes permitem empregar os processos da grande industria e
sucumbem na sua concorrência com os grandes capitalistas; outros;
porque a sua habilidade técnica se vê depreciada pelos novos métodos
de produção. De modo que o proletariado se recruta entre todas as
camadas da população.

O proletariado passa por diferentes etapas de desenvolvimento. A sua
luta contra a burguesia começa com a sua própria existência.

A princípio, a luta é entabulada por operários isolados, depois, por
operários de uma mesma fábrica, mais tarde, pelos operários do mesmo
ramo da indústria, numa mesma localidade, contra o burguês que os
explora diretamente. Não se contentam com dirigir os seus ataques
contra as relações burguesas de produção, e dirigem-se contra os
próprios instrumentos de produção: destroem as mercadorias
estrangeiras que lhes fazem concorrência, quebram as máquinas,
incendeiam as fábricas, tentam reconquistar pela força a posição
perdida do artesão da Idade Média.

Nesta etapa, os operários formam uma camada disseminada por todo o
país e desagregada pela concorrência. Se acontece que os operários se
apoiam pela ação da massa, esta ação não é ainda conseqüência da sua
própria unidade, mas da unidade da burguesia que, para alcançar os
seus próprios fins políticos, tem de pôr em movimento todo o
proletariado - e ainda possui, provisoriamente, o poder de o
fazer. Durante esta fase, os proletários não combatem, portanto,
contra os seus próprios inimigos, mas contra os inimigos dos seu
inimigos, quer dizer, contra os vestígios da monarquia absoluta, os
proprietários de terra, os burgueses não-industriais e os pequenos
burgueses. Todo o movimento histórico se concentra, deste modo, nas
mãos da burguesia; toda a vitória alcançada nestas condições é uma
vitória da burguesia.

Mas a industria, no seu desenvolvimento, não só aumenta o número de
proletários, como os concentra em massas consideráveis; a força dos
proletários aumenta e eles adquirem uma maior consciência dessa
força. Os interesses e as condições de existência dos proletários
igualam-se cada vez mais à medida que a máquina apaga as diferenças e
reduz o salário, quase em toda a parte, a um nível igualmente
baixo. Como resultado da crescente concorrência dos burgueses entre si
e das crises comerciais que daí resultam, os salários tornam-se cada
vez mais instáveis; o constante e acelerado aperfeiçoamento da máquina
coloca o operário numa situação cada vez mais precária; as colisões
individuais entre o operário e o burguês tomam cada vez mais o caráter
de colisões entre duas classes. Os operários começam por formar
coalizões contra os burgueses para a defesa dos seus
salários. Chegam a formar associações permanentes para assegurar os
meios necessários, na perspectiva de eventuais rebeliões. Aqui e além,
a luta rebenta, sob a forma de sublevações.

Por vezes, os operários triunfam; mas é um triunfo efêmero. O
verdadeiro resultado das suas lutas é menos o sucesso imediato do que
a união crescente dos trabalhadores. Esta união é favorecida pelo
crescimento dos meios de comunicação que são criados pela grande
indústria e que permitem aos operários de localidades diferentes
contatarem entre si. Ora, basta esse contato para que as numerosas
lutas locais, que por toda a parte revestem o mesmo caráter, se
centralizem numa luta nacional, numa luta de classes. Mas toda a luta
de classes é uma luta política, e a união que os burgueses da Idade
Média demoraram séculos a estabelecer através dos seus caminhos
vicinais, os proletários modernos realizam-na em poucos anos graças
aos caminhos de ferro.

Esta organização do proletariado em classe, e portanto em partido
político, é sem cessar socavada pela concorrência entre os próprios
operários. Mas renasce sempre, e cada vez mais forte, mais firme, mais
potente. Aproveita as divisões intestinas da burguesia para obrigar a
reconhecer por lei alguns interesses da classe operária: por exemplo o
bill da jornada de dez horas na Inglaterra.

Em geral, as colisões que se produzem na velha sociedade favorecem de
diversas maneiras o desenvolvimento do proletariado. A burguesia vive
num estado de guerra permanente: primeiro, contra a aristocracia,
depois, contra aquelas frações da mesma burguesia cujos interesses
entram em contradição com o progresso da indústria, e sempre,
finalmente, contra a burguesia de todos os países estrangeiros. Em
todas estas lutas, vê-se forçada a apelar para o proletariado, a
reclamar a sua ajuda e a arrastá-lo assim para o movimento
político. Deste modo, a burguesia proporciona aos proletários os
elementos da sua própria educação, isto é, armas contra ela própria.

Além disso, como acabamos de ver, o progresso da indústria precipita
nas fileiras do proletariado camadas inteiras da classe dominante, ou,
pelo menos, ameaça-as nas suas condições de existência. Também elas
trazem ao proletariado numerosos elementos de educação.

Finalmente, nos períodos em que a luta de classes se aproxima da hora
decisiva, o processo de desintegração da classe dominante, de toda a
velha sociedade, adquire um caráter tão violento e tão patente que uma
pequena fração da classe dominante renega esta e adere à classe
revolucionária, à classe que tem nas mãos o provir. E assim como,
outrora, uma parte da nobreza passou para a burguesia, nos nossos
dias, um setor da burguesia passa para o proletariado, particularmente
esse setor dos ideólogos burgueses que atingiram a compreensão teórica
do conjunto do movimento histórico.

De todas as classes que, na hora atual, se opõem à burguesia, só o
proletariado é uma classe verdadeiramente revolucionária. As outras
classes periclitam e perecem com o desenvolvimento da grande
indústria; o proletariado, pelo contrário, é o seu produto mais
autêntico.

As classes médias - o pequeno industrial, o pequeno comerciante, o
artesão, o camponês - todas combatem a burguesia porque ela é uma
ameaça para a sua existência como classes médias. Não são pois,
revolucionárias mas conservadoras. Mais ainda, são reacionárias, já
que pretendem fazer andar para trás a roda da história. São
revolucionárias unicamente quando têm diante de si a perspectiva da
sua passagem iminente ao proletariado: então, elas defendem os seus
interesses futuros e não os seus interesses atuais; abandonam o seu
próprio ponto de vista para adotar o do proletariado.

O lumpen-proletariado, esse produto passivo da putrefação das camadas
mais baixas da velha sociedade, pode por vezes ser arrastado para o
movimento por uma revolução proletária; no entanto, as condições de
vida dispô-lo-ão antes a vender-se à reação para servir as suas
manobras.

As condições de existência da velha sociedade estão já abolidas nas
condições de existência do proletariado. O proletariado não tem
propriedade; as suas relações com a mulher e com os filhos não têm
nada de comum com as da família burguesa; o trabalho industrial
moderno, a sujeição do operário ao capital, tanto na Inglaterra como
na França, na América do Norte como na Alemanha, despoja o
proletariado de todo o caráter nacional. As leis, a moral, a religião
são para os seus olhos outros tantos preconceitos burgueses, por
detrás dos quais se escondem outros tantos interesses burgueses.

Todas as classes que, no passado, se apoderaram do poder tentavam
consolidar a sua situação adquirida submetendo a sociedade às
condições do seu modo de apropriação. Os proletários não podem
conquistar as forças produtivas sociais, senão abolindo o seu próprio
modo de apropriação em vigor, e, por conseguinte, todo o modo de
apropriação existente até aos nossos dias. Os proletários não têm nada
a salvaguardar; têm que destruir tudo o que até agora vem garantindo e
assegurando a propriedade privada existente.

Todos os movimentos históricos foram até agora realizados por minorias
ou em proveito de minorias. O movimento proletário é o movimento
independente da imensa maioria em proveito da imensa maioria. O
proletariado, camada inferior da sociedade atual, não pode
levantar-se, não pode revoltar-se sem fazer saltar toda a
superestrutura das camadas que constituem a sociedade oficial.

A luta do proletariado contra a burguesia, ainda que não seja, pelo
seu conteúdo, uma luta nacional, reveste no entanto, inicialmente essa
forma. É evidente que o proletariado de cada país tem de acabar, antes
de mais, com a sua própria burguesia.

Ao esboçar em traços gerais as fases do desenvolvimento do
proletariado, descrevemos a história da guerra civil, mais ou menos
oculta, que se desenvolve no seio da sociedade existente, até ao
momento em que esta guerra se transforma numa revolução aberta e o
proletariado, derrubando pela violência a burguesia, implanta a sua
dominação.

Como vimos, todas as sociedades anteriores assentavam no antagonismo
entre classes opressoras e classes oprimidas. Mas para oprimir uma
classe, é preciso poder garantir-lhe condições de existência que lhe
permitam, pelo menos, viver na servidão. O servo, em pleno regime de
servidão, conseguiu tornar-se membro da comuna, do mesmo modo que o
pequeno burguês conseguiu elevar-se à categoria de burguês, sob o jugo
do absolutismo feudal. O operário moderno, pelo contrário, longe de se
elevar com o progresso da indústria, desce sempre mais e mais, abaixo
mesmo das condições de vida da sua própria classe. O trabalhador cai
na miséria, e o pauperismo cresce ainda mais rapidamente do que a
produção e a riqueza. É portanto manifesto que a burguesia é incapaz
de continuar a desempenhar por mais tempo o papel de classe dominante
na sociedade e de impor a esta, como lei reguladora, as condições de
existência da sua classe. Já não é capaz de reinar, porque não pode
assegurar ao escravo a existência, nem sequer dentro dos limites da
escravidão, porque é obrigada a deixa-lo decair até ao ponto de ter
que o manter, em vez de ter que ser mantida por ele. A sociedade já
não pode viver sob a sua dominação, o que equivale a dizer que a
existência da burguesia já não é compatível com a sociedade.

A condição essencial da existência e da dominação da classe burguesa é
a acumulação da riqueza nas mãos de particulares, a formação e o
crescimento do Capital. A condição de existência do Capital é o
trabalho assalariado. O trabalho assalariado assenta exclusivamente na
concorrência dos operários entre si. O progresso da indústria, de que
a burguesia, incapaz de se lhe opor, é agente involuntário, substitui
o isolamento dos operários, resultante da concorrência, pela sua união
revolucionária mediante a associação. Assim, o desenvolvimento da
grande indústria mina sob os pés da burguesia as bases sobre as quais
ela estabeleceu o sistema de produção e de apropriação. A burguesia
produz, antes de mais, os seus próprios coveiros. A sua queda e a
vitória do proletariado são igualmente inevitáveis.

\espaco{0,25cm}

\negrito{Proletários e Comunistas}

\espaco{0,25cm}

Qual é a posição dos comunistas em relação ao conjunto dos
proletários?

Os comunistas não formam um partido distinto, oposto aos outros
partidos operários.

Não têm interesses alguns que não sejam os interesses do conjunto do
proletariado.

Não proclamam princípios especiais sobre os quais queiram modelar o
movimento operário.

Os comunistas só se distinguem dos outros partidos operários em dois
pontos:

1. Nas diferentes lutas nacionais dos proletários, destacam e fazem
valer os interesses independentes da nacionalidade e comuns a todo o
proletariado;

2. Nas diferentes fases por que passa a luta entre proletários e
burgueses, representam sempre os interesses do movimento no seu
conjunto.

Praticamente, os comunistas são, pois, o setor mais resoluto dos
partidos operários de todos os países, têm sobre o resto do
proletariado a vantagem de uma clara compreensão das condições, da
marcha e dos fins gerais do movimento proletário.

O objetivo imediato dos comunistas é o mesmo que o de todos os outros
partidos proletários: constituição dos proletários em classe,
derrubamento da dominação burguesa, conquista do poder político pelo
proletariado.

As concepções teóricas dos comunistas não se baseiam de modo algum em
idéias e princípios inventados ou descobertos por este ou aquele
reformador do mundo.

Elas não são mais do que a expressão geral das condições reais de uma
luta de classes existente, de um movimento histórico que se desenvolve
diante dos nossos olhos. A abolição das relações de propriedade até
aqui existentes não é uma característica peculiar e exclusiva do
comunismo.

Todas as relações de propriedade sofreram constantes mudanças
históricas, contínuas transformações históricas.

A Revolução Francesa, por exemplo, aboliu a propriedade feudal em
proveito da propriedade burguesa.

O que caracteriza o comunismo não é a abolição da propriedade em
geral, mas a abolição da propriedade burguesa.

Ora, a propriedade privada de hoje, a propriedade burguesa, é a última
e mais acabada expressão do modo de produção e de apropriação baseado
nos antagonismos de classes, na exploração de uns pelos outros.

Neste sentido, os comunistas podem resumir a sua teoria a esta fórmula
única: abolição da propriedade privada.

Censuram-nos, a nós, comunistas, por querer abolir a propriedade
pessoalmente adquirida, fruto do trabalho do indivíduo, essa
propriedade que declaram ser a base de toda a liberdade, de toda a
atividade, de toda a independência individual.

A propriedade bem adquirida, fruto do trabalho, do esforço pessoal!
Referis-vos, por acaso, à propriedade do pequeno burguês, do pequeno
camponês, a essa forma de propriedade que precede a propriedade
burguesa? Não precisamos de aboli-la: o progresso da indústria
aboliu-a e continua a aboli-la diariamente.

Ou referi-vos talvez à propriedade privada moderna, à propriedade
burguesa?

Mas, será que o trabalho assalariado, o trabalho do proletário, cria
propriedade para o proletário? De maneira alguma. Ele cria o capital,
quer dizer, a propriedade que explora o trabalho assalariado e que só
pode acrescentar-se na condição de produzir mais e mais trabalho
assalariado, a fim de o explorar de novo. Na sua forma atual, a
propriedade move-se no antagonismo entre o capital e o trabalho
assalariado. Examinemos os dois termos deste antagonismo.

Ser capitalista significa ocupar não só uma posição meramente pessoal
na produção, mas também uma posição social. O capital é um produto
coletivo: só pode ser posto em movimento pela atividade conjunta de
todos os membros da sociedade.

O capital não é, pois, uma força pessoal; é uma força social.

Em conseqüência, se o capital se transforma em propriedade coletiva,
pertencente a todos os membros da sociedade, não é a propriedade
pessoal que se transforma em propriedade social. Só terá mudado o
caráter social da propriedade. Esta perderá o seu caráter de classe.

Examinemos o trabalho assalariado.

O preço médio do trabalho assalariado é o mínimo do salário, quer
dizer, a soma dos meios de subsistência indispensáveis ao operário
para manter a sua vida, como operário. Por conseguinte, aquilo de que
o operário se apropria pela sua atividade é o estritamente necessário
para reproduzir a sua vida, reduzida à sua simples expressão. Não
queremos de maneira nenhuma abolir esta apropriação pessoal dos
produtos do trabalho, indispensável à mera reprodução da vida humana,
essa apropriação que não deixa nenhum lucro líquido que confira um
poder sobre o trabalho de outrém. O que queremos suprimir é o caráter
miserável desta apropriação, que faz com que o operário não viva senão
para acrescentar o capital e tão só na medida em que o interesse da
classe dominante exige que viva.

Na sociedade burguesa, o trabalho vivo não é mais do que um meio para
aumentar o trabalho acumulado. Na sociedade comunista, o trabalho
acumulado não é mais do que um meio de ampliar, enriquecer e tornar
mais fácil a existência dos trabalhadores.

Deste modo, na sociedade burguesa, o passado domina o presente; na
sociedade comunista é o presente que domina o passado. Na sociedade
burguesa, o capital é independente e tem personalidade, enquanto que o
indivíduo que trabalha não tem independência, nem personalidade.

E é a abolição de semelhante estado de coisas o que a burguesia
considera como a abolição da personalidade e da liberdade! E com
razão. Pois trata-se efetivamente de abolir a personalidade burguesa,
a independência burguesa e a liberdade burguesa, entende-se a
liberdade de comércio, a liberdade de comprar e vender.

Mas se o tráfico desaparece, a liberdade de traficar desaparece
também. De resto, todas as grandes palavras sobre a liberdade de
comércio, do mesmo modo que as fanfarronadas liberais da nossa
burguesia, só têm sentido quando aplicadas ao tráfico entravado e ao
burguês subjugado da Idade Média; mas não têm nenhum sentido quando se
trata da abolição, pelo comunismo, do tráfico, das relações de
produção burguesas e da própria burguesia.

Ficais horrorizados por querermos abolir a propriedade privada. Mas na
vossa sociedade atual a propriedade privada está abolida para nove
décimos dos seus membros. É precisamente porque não existe para esses
nove décimos que ele existe para vós. Reprovai-nos, pois, o querer
abolir uma forma de propriedade que só pode existir na condição da
imensa maioria da sociedade ser privada de qualquer propriedade.

Numa palavra, acusais-nos de querer abolir a vossa propriedade. Na
verdade, é isso que queremos.

Segundo vós, a partir do momento em que o trabalho não pode ser
convertido em capital, em dinheiro, em renda da terra, numa palavra,
em poder social susceptível de ser monopolizado; quer dizer, a partir
do momento em que a propriedade pessoal não pode transformar-se em
propriedade burguesa, a partir desse momento a personalidade fica
suprimida.

Confessais, pois, que por personalidade só entendeis o burguês, o
proprietário burguês. E essa personalidade deve, certamente, ser
suprimida.

O comunismo não tira a ninguém a faculdade de se apropriar dos
produtos sociais; ele não tira mais do que o poder de subjugar o
trabalho alheio por meio desta apropriação.

Objetou-se ainda que, com a abolição da propriedade privada cessaria
toda a atividade e uma preguiça geral se apoderaria do mundo.

Se assim fosse, já há muito tempo que a sociedade burguesa teria
sucumbido à ociosidade, visto que, nesta sociedade, os que trabalham
não ganham e os que ganham não trabalham.

Toda a objeção se reduz a esta tautologia: onde não há capital, não há
trabalho assalariado.

Todas as objeções dirigidas contra o modo comunista de apropriação e
de produção dos elementos materiais foram igualmente feitas em relação
à apropriação e à produção dos produtos do trabalho intelectual. Do
mesmo modo que, para o burguês, o desaparecimento da propriedade de
classe equivale ao desaparecimento de toda a produção, o
desaparecimento da cultura de classe significa para ele o
desaparecimento de toda a cultura.

A cultura, cuja perda deplora, não é mais, para a imensa maioria dos
homens, do que o adestramento que os transforma em máquinas.

Mas é inútil procurar discutir conosco enquanto aplicardes à abolição
da propriedade burguesa o critério das vossas noções burguesas de
liberdade, cultura, direito, etc. As vossas idéias são, em si mesmas,
produto das relações de produção e de propriedade burguesas, assim
como o vosso direito não é mais do que a vontade da vossa classe
erigida em lei; vontade cujo conteúdo está determinado pelas condições
materiais de existência da vossa classe.

A concepção interessada que vos fez erigir em leis eternas da Natureza
e da Razão as relações sociais emanadas do vosso transitório modo de
produção e de propriedade - relações históricas que surgem e
desaparecem no curso da produção - partilha-la como todas as classes
dominantes hoje desaparecidas. O que concebeis para a propriedade
antiga, o que concebeis para a propriedade feudal, não vos atreveis a
admiti-lo para a propriedade burguesa.

Querer abolir a família! até os mais radicais se indignam perante este
infame desígnio dos comunistas.

Em que base assenta a família atual, a família burguesa? No capital,
no lucro privado. A família, plenamente desenvolvida, não existe, a
não ser para a burguesia; mas ela tem por corolário a supressão
forçada de toda a família para o proletariado e a prostituição
pública.

A família burguesa desaparece naturalmente ao deixar de existir o seu
corolário, e um e outro desaparecem com o desaparecimento do capital.

Reprovais-nos o querer abolir a exploração dos filhos pelos pais?
Confessamos esse crime.

Mas dizeis que destruímos os vínculos mais íntimos, substituindo a
educação em família pela educação social.

E a vossa educação, não está também determinada pela sociedade, pelas
condições sociais em que educais os vossos filhos, pela intervenção
direta ou indireta da sociedade através da escola, etc.? Os comunistas
não inventaram esta ingerência da sociedade na educação, eles não
fazem mais do que mudar o seu caracter e arrancar a educação à
influência da classe dominante.

As declamações burguesas sobre a família e a educação, sobre os doces
laços que unem os filhos aos pais, tornam-se cada vez mais repugnantes
à medida que a grande indústria destrói todos os vínculos de família
para o proletário e transforma as crianças em simples artigos de
comércio, em simples instrumentos de trabalho.

Mas vós, os comunistas, quereis estabelecer a comunidade das
mulheres!---gritam-nos, em coro, toda a burguesia.

Para o burguês, a sua mulher não é mais do que um instrumento de
produção. Ouve dizer que os instrumentos de produção devem ser
utilizados em comum, e, naturalmente, não pode deixar de pensar que as
mulheres partilharão a sorte comum da socialização.

Não suspeita que se trata precisamente de arrancar a mulher ao seu
papel atual de simples instrumento de produção.

Nada mais grotesco, aliás, do que o horror ultramoral que inspira aos
nossos burgueses a pretensa comunidade oficial das mulheres que
atribuem aos comunistas. Os comunistas não têm necessidade de
introduzir a comunidade das mulheres: ela existiu quase sempre.

Os nossos burgueses, não satisfeitos de ter à sua disposição as
mulheres e as filhas dos proletários, sem falar da prostituição
oficial, encontram um prazer singular em cornear-se mutuamente.

O matrimônio burguês é, na realidade, a comunidade das esposas. Quando
muito, poder-se-ia acusar os comunistas de querer substituir uma
comunidade de mulheres hipocritamente dissimulada, por uma comunidade
franca e oficial. É evidente, de resto, que, com a abolição das
relações de produção atuais desaparecerá a comunidade das mulheres que
delas deriva, quer dizer, a prostituição oficial e não-oficial.

Acusam-se também os comunistas de quer abolir a pátria, a
nacionalidade.

Os operários não têm pátria. Não se lhes pode tirar aquilo que não
possuem. Mas, como o proletariado tem, em primeiro lugar, de
conquistar o poder político, elevar-se à condição de classe
nacional, constituir-se em nação, é ainda nacional, posto que, de
maneira nenhuma, no sentido burguês.

O isolamento nacional e os antagonismos entre os povos desaparecem de
dia para dia com o desenvolvimento da burguesia, a liberdade de
comércio e o mercado mundial, com a uniformidade da produção
industrial e as condições de existência que lhe correspondem.

O proletariado no poder os fará desaparecer ainda mais depressa. A
ação comum do proletariado, pelo menos nos países civilizados, é uma
das primeiras condições da sua emancipação.

Aboli a exploração do homem pelo homem, e abolireis a exploração de
uma nação por outra nação.

Ao mesmo tempo que o antagonismo das classes no interior das nações,
desaparecerá a hostilidade das nações entre si.

Quanto às acusações lançadas, dum modo geral, contra o comunismo,
partindo de pontos de vista religiosos, filosóficos e ideológicos, não
merecem um exame aprofundado.

Será necessária uma grande perspicácia para compreender que as ideias,
as concepções e as noções dos homens, numa palavra, a sua consciência,
mudam com toda a mudança sobrevinda nas suas condições de vida, nas
suas relações sociais, na sua existência social?

Que demonstra a história das idéias senão que a produção intelectual
se transforma com a produção material? As idéias dominantes em
qualquer época nunca passaram das idéias da classe dominante.

Quando se fala de idéias que revolucionam toda uma sociedade,
exprime-se apenas o fato de que no seio da velha sociedade se formaram
os elementos de uma sociedade nova, e de que a dissolução das velhas
idéias marcha a par da dissolução das antigas condições de existência.

Quando o mundo antigo estava no seu declínio, as velhas religiões
foram vencidas pela religião cristã. Quando no século XVIII, as ideias
cristãs foram vencidas pelas idéias do iluminismo, a sociedade feudal
travava uma luta de morte contra a burguesia, então revolucionária. As
idéias de liberdade religiosa e de liberdade de consciência não
fizeram mais do que refletir o reinado da livre concorrência no
domínio da consciência.23

Sem dúvida---dir-se-á---as ideias religiosas, morais, filosóficas,
políticas, jurídicas, etc., modificaram-se no decurso do processo
histórico. Mas a religião, a moral, a filosofia, a política, o direito
mantiveram-se sempre através dessas transformações.

Existem, além disso, verdades eternas, tais como a liberdade, a
justiça, etc., que são comuns, a todos os regimes sociais. Mas o
comunismo quer abolir estas verdades eternas, quer abolir a religião e
a moral, em vez de dar-lhe uma forma nova, e isso contradiz todo o
processo histórico anterior".

A que se reduz esta acusação? A história de todas as sociedades que
existiram até hoje era feita de antagonismos de classes, de
antagonismos que revestem formas diversas nas diferentes épocas.

Mas qualquer que tenha sido a forma destes antagonismos, a exploração
de uma parte da sociedade pela outra é um fato comum a todos os
séculos anteriores. Por conseguinte, não é de espantar que a
consciência social de todos os séculos, a despeito de toda a variedade
e de toda a diversidade, se tenha movido sempre dentro de certas
formas comuns, dentro de umas formas24 - formas de consciência - que
só desaparecerão completamente com o desaparecimento definitivo dos
antagonismos de classes.

A revolução comunista é a ruptura mais radical com as relações de
propriedade tradicionais, portanto, não há nada de estranho em que no
decurso do seu desenvolvimento rompa da maneira mais radical com as
idéias tradicionais.

Mas, deixemos as objeções feitas pela burguesia ao comunismo.

Como já vimos mais acima, o primeiro passo da revolução operária é a
elevação do proletariado a classe dominante, a conquista da
democracia.

O proletariado servi-se-á da sua supremacia política para arrancar
pouco a pouco à burguesia todo o capital, para centralizar todos os
instrumentos de produção nas mãos do Estado, quer dizer, do
proletariado organizado como classe dominante, e para aumentar com a
maior rapidez possível a quantidade das forças produtivas.

Isto naturalmente, não poderá fazer-se, de inicio, senão por uma
violação despótica de direito de propriedade e das relações burguesas
de produção, quer dizer, pela adoção de medidas que do ponto de vista
econômico, parecem insuficientes e insustentáveis, mas que, no decurso
do movimento ultrapassar-se-ão a si mesmas25 e serão indispensáveis
como meio de transformar radicalmente todo o modo de produção.

Estas medidas, naturalmente, serão muito diferentes nos diversos
países.

No entanto, nos países mais avançados poderão ser postas em prática
quase em toda a parte as seguintes medidas:

1. Expropriação da propriedade da terra e afetação da renda da terra
às despesas do Estado.

2. Imposto fortemente progressivo.

3. Abolição do direito de herança.

4. Confiscação da propriedade de todos os emigrados e sediciosos.

5. Centralização do crédito nas mãos do Estado, por meio de um Banco
nacional, com capital do Estado e monopólio exclusivo.

6. Centralização nas mãos do Estado de todos os meios de transporte.

7. Multiplicação das empresas fabris pertencentes ao Estado e dos
instrumentos de produção, arroteamento dos terrenos incultos e
melhoramento das terra cultivadas, segundo um plano se conjunto.

8. Trabalho obrigatório para todos; organização de exércitos
industriais, particularmente para a agricultura.

9. Combinação da agricultura e da industria; medidas tendentes a fazer
desaparecer gradualmente o antagonismo26 entre a cidade e o campo.27

10. Educação pública e gratuita de todas as crianças; abolição do
trabalho das crianças nas fábricas tal como hoje se
pratica. Combinação da educação com a produção material, etc.

Uma vez que no decurso do desenvolvimento tiverem desaparecido os
antagonismos de classe e se tiver concentrado toda a produção nas mãos
de indivíduos associados, o poder publico perderá o seu caráter
político. O poder político, para falar com propriedade, é a violência
organizada de uma classe dominante, destrói pela violência as antigas
relações de produção, suprime ao mesmo tempo que estas relações de
produção as condições para a existência do antagonismo das classes em
geral,28 e, portanto, a sua própria dominação como classe.

Em substituição da antiga sociedade burguesa, com as suas classes e os
seus antagonismos de classe, surgirá uma associação em que o livre
desenvolvimento de cada um será a condição do livre desenvolvimento de
todos.

\fim
