% Nossos códigos de categoria são iguais aos do Plain TeX, exceto por
% não suportarmos os caracteres ASCII de seta para cima e para baixo
% como sobrescrito e subscrito. Atualmente não há teclados com tais
% símbolos sendo encontrados por aí.

\catcode`\{=1
\catcode`\}=2
\catcode`\$=3
\catcode`\&=4
\catcode`\#=6
\catcode`\^=7
\catcode`\_=8
\catcode`\^^I=10
\catcode`\~=13

% Assim como na definição do Plain TeX, vamos reservar o @ para atuar
% temporariamente como letra e assim podermos reservar comandos
% privados que terão '@' no nome e que não poderão ser geralmente
% sobrescritos ou usados por um usuário. Depois faremos com que o @
% volte a ser uma marca de pontuação.
\catcode`@=11

% Usamos as mesmas configurações do fator de espaço que o Plain TeX. O
% espaço após o ')' é igual ao espaço do que vem antes dele, e o mesmo
% ocorre com aspas e fechamento de colchetes. Adotamos por padrão as
% convenções do \nonfrenchspacing
\sfcode`\)=0 \sfcode`\'=0 \sfcode`\]=0 \sfcode`\.3000 \sfcode`\?3000
\sfcode`\!3000 \sfcode`\:2000 \sfcode`\;1500 \sfcode`\,1250

% O Plain TeX não faz modificações na tabela padrão de conversão de
% caracteres maiúsculos e minúsculos. Mas aqui é necessário
% adicionarmos novos itens à tabela para que possamos converter para
% maiúscula e minúscula os caracteres UTF-8 que representam letras
% acentuadas. Todos eles começam com o byte ^^c3 de prefixo que nunca
% muda. O caractere seguinte define qual caractere unicode ele é.
\lccode`^^c3=`^^c3\uccode`^^c3=`^^c3 % Prefixo
\lccode`^^80=`^^a0\uccode`^^a0=`^^80\lccode`^^a0=`^^a0\uccode`^^80=`^^80 % À - à
\lccode`^^81=`^^a1\uccode`^^a1=`^^81\lccode`^^a1=`^^a1\uccode`^^81=`^^81 % Á - á
\lccode`^^82=`^^a2\uccode`^^a2=`^^82\lccode`^^a2=`^^a2\uccode`^^82=`^^82 % Â - â
\lccode`^^83=`^^a3\uccode`^^a3=`^^83\lccode`^^a3=`^^a3\uccode`^^83=`^^83 % Ã - ã
\lccode`^^84=`^^a4\uccode`^^a4=`^^84\lccode`^^a4=`^^a4\uccode`^^84=`^^84 % Ä - ä
\lccode`^^85=`^^a5\uccode`^^a5=`^^85\lccode`^^a5=`^^a5\uccode`^^85=`^^85 % AA-aa
\lccode`^^86=`^^a6\uccode`^^a6=`^^86\lccode`^^a6=`^^a6\uccode`^^86=`^^86 % AE-ae
\lccode`^^87=`^^a7\uccode`^^a7=`^^87\lccode`^^a7=`^^a7\uccode`^^87=`^^87 % Ç - ç
\lccode`^^88=`^^a8\uccode`^^a8=`^^88\lccode`^^a8=`^^a8\uccode`^^88=`^^88 % È - è
\lccode`^^89=`^^a9\uccode`^^a9=`^^89\lccode`^^a9=`^^a9\uccode`^^89=`^^89 % É - é
\lccode`^^8a=`^^aa\uccode`^^aa=`^^8a\lccode`^^aa=`^^aa\uccode`^^8a=`^^8a % Ê - ê
\lccode`^^8b=`^^ab\uccode`^^ab=`^^8b\lccode`^^ab=`^^ab\uccode`^^8b=`^^8b % Ë - ë
\lccode`^^8c=`^^ac\uccode`^^ac=`^^8c\lccode`^^ac=`^^ac\uccode`^^8c=`^^8c % Ì - ì
\lccode`^^8d=`^^ad\uccode`^^ad=`^^8d\lccode`^^ad=`^^ad\uccode`^^8d=`^^8d % Í - í
\lccode`^^8e=`^^ae\uccode`^^ae=`^^8e\lccode`^^ae=`^^ae\uccode`^^8e=`^^8e % Î - î
\lccode`^^8f=`^^af\uccode`^^af=`^^8f\lccode`^^af=`^^af\uccode`^^8f=`^^8f % Ï - ï
\lccode`^^91=`^^b1\uccode`^^b1=`^^91\lccode`^^b1=`^^b1\uccode`^^91=`^^91 % Ñ - ñ
\lccode`^^92=`^^b2\uccode`^^b2=`^^92\lccode`^^b2=`^^b2\uccode`^^92=`^^92 % Ò - ò
\lccode`^^93=`^^b3\uccode`^^b3=`^^93\lccode`^^b3=`^^b3\uccode`^^93=`^^93 % Ó - ó
\lccode`^^94=`^^b4\uccode`^^b4=`^^94\lccode`^^b4=`^^b4\uccode`^^94=`^^94 % Ô - ô
\lccode`^^95=`^^b5\uccode`^^b5=`^^95\lccode`^^b5=`^^b5\uccode`^^95=`^^95 % Õ - õ
\lccode`^^96=`^^b6\uccode`^^b6=`^^96\lccode`^^b6=`^^b6\uccode`^^96=`^^96 % Ö - ö
\lccode`^^98=`^^b8\uccode`^^b8=`^^98\lccode`^^b8=`^^b8\uccode`^^98=`^^98 % /O-/o
\lccode`^^99=`^^b9\uccode`^^b9=`^^99\lccode`^^b9=`^^b9\uccode`^^99=`^^99 % Ù - ù
\lccode`^^9a=`^^ba\uccode`^^ba=`^^9a\lccode`^^ba=`^^ba\uccode`^^9a=`^^9a % Ú - ú
\lccode`^^9b=`^^bb\uccode`^^bb=`^^9b\lccode`^^bb=`^^bb\uccode`^^9b=`^^9b % Û - û
\lccode`^^9c=`^^bc\uccode`^^bc=`^^9c\lccode`^^bc=`^^bc\uccode`^^9c=`^^9c % Ü - ü
\lccode`^^9d=`^^bd\uccode`^^bd=`^^9d\lccode`^^bd=`^^bd\uccode`^^9d=`^^9d % Ý - ý

% Código que passa os padrões de separação silábica da língua
% portuguesa:
\patterns{1ba1 1be1 1bi1 1bo1 1bu1 1ca1 1ce1 1ci1 1co1 1cu1
          1da1 1de1 1di1 1do1 1du1 1fa1 1fe1 1fi1 1fo1 1fu1
          1ga1 1ge1 1gi1 1go1 1gu1 1ha1 1he1 1hi1 1ho1 1hu1
          1ja1 1je1 1ji1 1jo1 1ju1 1ka1 1ke1 1ki1 1ko1 1ku1
          1la1 1le1 1li1 1lo1 1lu1 1ma1 1me1 1mi1 1mo1 1mu1
          1na1 1ne1 1ni1 1no1 1nu1 1pa1 1pe1 1pi1 1po1 1pu1
          1qua1 1que1 1qui1 1quo1 1ra1 1re1 1ri1 1ro1 1ru1
          1sa1 1se1 1si1 1so1 1su1 1ta1 1te1 1ti1 1to1 1tu1
          1va1 1ve1 1vi1 1vo1 1vu1 1wa1 1we1 1wi1 1wo1 1wu1
          1xa1 1xe1 1xi1 1xo1 1xu1 1za1 1ze1 1zi1 1zo1 1zu1}

% Código para suportar os caracteres Computer Modern na fonte
% UTF-8. Primeiro fazemos com que ^^c3 passe a ser um caractere ativo
% que passa a se comportar como um comando (tal como o ~). Este
% caractere é o prefixo de tudo quanto é letra acentuada e Cs com
% cedilhas. Em seguida, definimos o ``comando'' ^^c3 para olhar pelo
% caractere seguinte e, baseado nele, inserir o caractere adequado.

%TODO: ç e Ç sem \c
\catcode`\^^c3=13
\def^^c3#1{\ifx#1^^80\ifmmode{\hbox{\tenit@\accent18A}}\else{\accent18A}\fi% À
        \else\ifx#1^^81\ifmmode{\acute A}\else{\accent19A}\fi% Á
        \else\ifx#1^^82\ifmmode{\hat A}\else{\accent94A}\fi% Â
        \else\ifx#1^^83\ifmmode{\tilde A}\else{\accent"7EA}\fi% Ã
        \else\ifx#1^^84\ifmmode{\ddot A}\else{\accent"7FA}\fi% Ä
        \else\ifx#1^^85\ifmmode{\hbox{\tenit@\AA}}\else{\AA}\fi
        \else\ifx#1^^86\ifmmode{\hbox{\tenit@\char"1D}}\else\char"1D\fi
        \else\ifx#1^^87\ifmmode\hbox{\tenit@\c C}\else\c C{}\fi % TODO
        \else\ifx#1^^88\ifmmode\hbox{\accent18E}\else{\accent18E}\fi%
        \else\ifx#1^^89\ifmmode{\acute E}\else{\accent19E}\fi% É
        \else\ifx#1^^8a\ifmmode{\hat E}\else{\accent94E}\fi% Ê
        \else\ifx#1^^8b\ifmmode{\ddot E}\else{\accent"7FE}\fi% Ë
        \else\ifx#1^^8c\ifmmode\hbox{\tenit@\accent18I}\else{\accent18I}\fi% Ì
        \else\ifx#1^^8d\ifmmode{\acute I}\else{\accent19I}\fi% Í
        \else\ifx#1^^8e\ifmmode{\hat I}\else{\accent94I}\fi% Î
        \else\ifx#1^^8f\ifmmode{\ddot I}\else{\accent"7FI}\fi% Ï
        \else\ifx#1^^91\ifmmode{\tilde N}\else{\accent"7EN}\fi% Ñ
        \else\ifx#1^^92\ifmmode\hbox{\tenit@\accent18O}\else{\accent18O}\fi% Ò
        \else\ifx#1^^93\ifmmode{\acute O}\else{\accent19O}\fi% Ó
        \else\ifx#1^^94\ifmmode{\hat O}\else{\accent94O}\fi% Ô
        \else\ifx#1^^95\ifmmode{\tilde O}\else{\accent"7EO}\fi% Õ
        \else\ifx#1^^96\ifmmode{\ddot O}\else{\accent"7FO}\fi% Ö
        \else\ifx#1^^98\ifmmode\hbox{\tenit@\char"1F}\else\char"1F\fi% \O
        \else\ifx#1^^99\ifmmode\hbox{\tenit@\accent18U}\else{\accent18U}\fi% Ù
        \else\ifx#1^^9a\ifmmode{\acute U}\else{\accent19U}\fi% Ú
        \else\ifx#1^^9b\ifmmode{\hat U}\else{\accent94U}\fi% Û
        \else\ifx#1^^9c\ifmmode{\ddot U}\else{\accent"7FU}\fi% Ü
        \else\ifx#1^^9d\ifmmode{\acute Y}\else{\accent19Y}\fi% Ý
        \else\ifx#1^^a0\ifmmode\hbox{\tenit@\accent18a}\else{\accent18a}\fi% à
        \else\ifx#1^^a1\ifmmode{\acute a}\else{\accent19a}\fi% á
        \else\ifx#1^^a2\ifmmode{\hat a}\else{\accent94a}\fi% â
        \else\ifx#1^^a3\ifmmode{\tilde a}\else{\accent"7Ea}\fi% ã
        \else\ifx#1^^a4\ifmmode{\ddot a}\else{\accent"7Fa}\fi% ä
        \else\ifx#1^^a5\ifmmode\hbot{\tenit@\aa}\else\aa{}\fi
        \else\ifx#1^^a6\ifmmode\hbox{\tenit@\char"1A}\else\char"1A\fi%
        \else\ifx#1^^a7\ifmmode\hbox{\tenit@\c c}\else\c c\fi
        \else\ifx#1^^a8\ifmmode\hbox{\tenit@\accent18e}\else{\accent18e}\fi% è
        \else\ifx#1^^a9\ifmmode{\acute e}\else{\accent19e}\fi% é
        \else\ifx#1^^aa\ifmmode{\hat e}\else{\accent94e}\fi% ê
        \else\ifx#1^^ab\ifmmode{\ddot e}\else{\accent"7Fe}\fi% ë
        \else\ifx#1^^ac\ifmmode\hbox{\tenit@\accent18\i}\else{\accent18\i}\fi% ì
        \else\ifx#1^^ad\ifmmode\hbox{\tenit@\accent19\i}\else{\accent19\i}\fi% í
        \else\ifx#1^^ae\ifmmode\hbox{\tenit@\accent94\i}\else{\accent94\i}\fi% î
        \else\ifx#1^^af\ifmmode\hbox{\tenit@\accent"7F\i}\else{\accent"7F\i}\fi% ï
        \else\ifx#1^^b1\ifmmode{\tilde n}\else{\accent"7En}\fi% ñ
        \else\ifx#1^^b2\ifmmode\hbox{\tenit@\accent18o}\else{\accent18o}\fi% ò
        \else\ifx#1^^b3\ifmmode{\acute o}\else{\accent19o}\fi% ó
        \else\ifx#1^^b4\ifmmode{\hat o}\else{\accent94o}\fi% ô
        \else\ifx#1^^b5\ifmmode{\tilde o}\else{\accent"7Eo}\fi% õ
        \else\ifx#1^^b6\ifmmode{\ddot o}\else{\accent"7Fo}\fi% ö
        \else\ifx#1^^b8\ifmmode\hbox{\tenit@\char"1C}\else\char"1C\fi% \o
        \else\ifx#1^^b9\ifmmode\hbox{\tenit@\accent18u}\else{\accent18u}\fi% ù
        \else\ifx#1^^ba\ifmmode{\acute u}\else{\accent19u}\fi% ú
        \else\ifx#1^^bb\ifmmode{\hat u}\else{\accent94u}\fi% û
        \else\ifx#1^^bc\ifmmode{\ddot u}\else{\accent"7Fu}\fi% ü
        \else\ifx#1^^bd\ifmmode{\acute y}\else{\accent19y}\fi% ý
        \else\ifx#1^^bf\ifmmode{\ddot y}\else{\accent"7FY}\fi% ÿ
  \fi\fi\fi\fi\fi\fi\fi\fi\fi\fi\fi\fi\fi\fi
  \fi\fi\fi\fi\fi\fi\fi\fi\fi\fi\fi\fi\fi\fi\fi\fi\fi\fi\fi\fi\fi\fi\fi\fi\fi\fi
  \fi\fi\fi\fi\fi\fi\fi\fi\fi\fi\fi\fi\fi\fi\fi\fi\fi}

% Permitindo inserir acentos soltos, sem caracteres por meio
% de \~, \^, etc. Estes comandos são diferentes do Plain TeX, pois não
% receberão argumentos:
\def\~{\accent"7E\ }\def\"{\accent"7F\ }\def\'{\accent19\ }
\def\^{\accent94\ }\def\`{\accent18\ }

% O comando para terminar um documento é \fim:
\def\fim{\par\vfill\supereject\end}

% Os códigos dos caracteres matemáticos não serão modificados. Como o
% modo matemático do TeX e LaTeX são em si mesmo uma linguagem própria
% já muito difundida, não pretendendo realizar nenhuma mudança nem
% aqui e nem em nenhum outro lugar referente ao modo matemático. Os
% caracteres definidos terão o mesmo código do Apêndice C do
% Texbook. Tal \mathcode é usado para definir o comportamento da
% tipografia de tais símbolos, a sua fonte e sua posição
% específica. Alguns caracteres como ' também passam a ser
% considerados caracteres especiais.
\mathcode`\^^@="2201 % \cdot
\mathcode`\^^A="3223 % \downarrow
\mathcode`\^^B="010B % \alpha
\mathcode`\^^C="010C % \beta
\mathcode`\^^D="225E % \land
\mathcode`\^^E="023A % \lnot
\mathcode`\^^F="3232 % \in
\mathcode`\^^G="0119 % \pi
\mathcode`\^^H="0115 % \lambda
\mathcode`\^^I="010D % \gamma
\mathcode`\^^J="010E % \delta
\mathcode`\^^K="3222 % \uparrow
\mathcode`\^^L="2206 % \pm
\mathcode`\^^M="2208 % \oplus
\mathcode`\^^N="0231 % \infty
\mathcode`\^^O="0140 % \partial
\mathcode`\^^P="321A % \subset
\mathcode`\^^Q="321B % \supset
\mathcode`\^^R="225C % \cap
\mathcode`\^^S="225B % \cup
\mathcode`\^^T="0238 % \forall
\mathcode`\^^U="0239 % \exists
\mathcode`\^^V="220A % \otimes
\mathcode`\^^W="3224 % \leftrightarrow
\mathcode`\^^X="3220 % \leftarrow
\mathcode`\^^Y="3221 % \rightarrow
\mathcode`\^^Z="8000 % \ne
\mathcode`\^^[="2205 % \diamond
\mathcode`\^^\="3214 % \le
\mathcode`\^^]="3215 % \ge
\mathcode`\^^^="3211 % \equiv
\mathcode`\^^_="225F % \lor
\mathcode`\ ="8000 % \space
\mathcode`\!="5021
\mathcode`\'="8000 % ^\prime
\mathcode`\(="4028
\mathcode`\)="5029
\mathcode`\*="2203 % \ast
\mathcode`\+="202B
\mathcode`\,="613B
\mathcode`\-="2200
\mathcode`\.="013A
\mathcode`\/="013D
\mathcode`\:="303A
\mathcode`\;="603B
\mathcode`\<="313C
\mathcode`\=="303D
\mathcode`\>="313E
\mathcode`\?="503F
\mathcode`\[="405B
\mathcode`\\="026E % \backslash
\mathcode`\]="505D
\mathcode`\_="8000 % \_
\mathcode`\{="4266
\mathcode`\|="026A
\mathcode`\}="5267
\mathcode`\^^?="1273 % \smallint

% Assim como Plain TeX, usamos exatamente as mesmas escolhas de
% caracteres como delimitadores em modo matemático. Tem também
% o \delcode'.=0 que já está implícito pelo INITEX:
\delcode`\(="028300
\delcode`\)="029301
\delcode`\[="05B302
\delcode`\]="05D303
\delcode`\<="26830A
\delcode`\>="26930B
\delcode`\/="02F30E
\delcode`\|="26A30C
\delcode`\\="26E30F

% magitex assume que registradores devem ser tratados apenas por
% escritores de formato e pessoas que sabem bem o que estão fazendo. E
% que cada formato deve documentar quais registradores usa. Comandos
% do Plain TeX como \newcount podem ser abusados e podem exaurir o
% espaço de registradores. Então será proposital não
% declararmos \newcount, \newskip e outros comandos para dar nomes à
% registradores sem especificar os números.

% Os registradores de contagem de 0 à 9 são reservados. O 0 é o número
% da página. Os demais só são usados em formatos que de alguma forma
% precisam numerar sub-páginas e não serão usados. Depois deles nós
% temos:
\countdef\pageno@=0 \pageno@=1 % O número de página
\countdef\interdisplaylinepenalty=10 \interdisplaylinepenalty=100
\countdef\interfootnotelinepenalty=11 \interfootnotelinepenalty=100
\countdef\mscount=12
\countdef\count@=13

% Não há nenhum registrador de dimensão alocado pelas primitivas
% TeX. Desta forma, temos todos à nossa disposição:
\dimendef\maxdimen=0 \maxdimen=16383.99999pt % A dimensão máxima
\dimendef\normallineskiplimit=1 \normallineskiplimit=0pt
\dimendef\p@=2 \p@=1pt % this saves macro space and time
\dimendef\z@=3 \z@=0pt % can be used both for 0pt and 0
\dimendef\jot=4 \jot=3pt
\dimendef\p@renwd=5
\dimendef\dimen@=6
\dimendef\dimen@i=7
\dimendef\dimen@ii=8


% Os registradores de skip. Nenhum é reservado e todos estão à
% disposição:
\skipdef\hideskip=0 \hideskip=-1000pt plus 1fill % negative but can grow
\skipdef\centering=1 \centering=0pt plus 1000pt minus 1000pt
\skipdef\z@skip=2 \z@skip=0pt plus0pt minus0pt
\skipdef\smallskipamount=3 \smallskipamount=3pt plus 1pt minus 1pt
\skipdef\medskipamount=4 \medskipamount=6pt plus 2pt minus 2pt
\skipdef\bigskipamount=5 \bigskipamount=12pt plus 4pt minus 4pt
\skipdef\normalbaselineskip=6 \normalbaselineskip=12pt
\skipdef\normallineskip=7 \normallineskip=1pt
\skipdef\skip@=8

% Nenhum registrador de muskip é usado

% Registradores de caixa estão todos à disposição, exceto o de número
% 255:
\def\voidb@x{0} % permanently void box register
\def\strutbox{1} \setbox\strutbox=\hbox{\vrule height8.5pt depth3.5pt
  width\z@}
\def\tabs{2}
\def\tabsyet{3}
\def\tabsdone{4}
\def\rootbox{5}

% Agora reservamos os registradores de sequências de tokens:
\toksdef\headline=0 \headline={\hfil} % Cabeçalho padrão em branco
\toksdef\footline=1 \footline={\hss\tenrm@\folio\hss} % Rodapé padrão

% Comandos de espaçamento
\def\espacito{\kern .16667em } % Equivalente ao \thinspace do Plain TeX
\def\espacinho{\vskip\smallskipamount}
\def\espaco{\vskip\medskipamount}
\def\espacao{\vskip\bigskipamount}

\def\espacoh#1{\hskip#1}
\def\espacov#1{\vskip#1}

% Comandos \TeX e \MaGiTeX:
\def\TeX{T\kern-.1667em\lower.5ex\hbox{E}\kern-.125emX}
\def\MaGiTeX{M\kern-0.15em\lower0.5ex\hbox{A}\kern-0.25emG\lower
        0.5ex\hbox{I}\kern-0.1em\TeX}

% Os nomes internos das fontes em magitex devem ser
% reservados. Somente projetistas de formatos deveriam lidar com tais
% modificações
\defaulthyphenchar=`\- % O caractere de separação silábica deve ser
                       % setado antes.
\font\tenit@=cmti10 % text italic
\font\tensl@=cmsl10 % slanted roman
\font\tenrm@=cmr10 % roman text (mudar o nome \tenrm gera problemas)
\font\tenbf@=cmbx10 % boldface extended
\font\tentt@=cmtt10 % typewriter
\font\teni@=cmmi10 % math italic
\font\tensy@=cmsy10 % math symbols
\font\tenex@=cmex10 % math extension
\font\sevenrm@=cmr7 % Usado em subscritos e sobrescritos
\font\fiverm@=cmr5 % Usado em subsubscritos e sobresobrescritos
\font\seveni@=cmmi7 % Sobrescritos e subscritos matemáticos
\font\fivei@=cmmi5 % Subsubscritos itálicos
\font\sevensy@=cmsy7 % Símbolos subscritos
\font\fivesy@=cmsy5 % Símbolos subsubscritos
\font\sevenbf@=cmbx7 % Subscritos negritos
\font\fivebf@=cmbx5 % Subsubscritos negritos

% Comandos de estilo de escrita. Colocando correção itálica
% automaticamente se necessário.
% TODO: #2 precisa ser mesmo opcional abaixo:
\long\def\italico#1#2{{\fam4\tenit@ #1\ifx#2.#2\else\ifx#2,#2\else\/#2\fi\fi}}
\long\def\romano#1{{\fam0\tenrm@ #1}}
\long\def\negrito#1{{\fam6\tenbf@ #1}}
\long\def\inclinado#1#2{{\fam5\tensl@ #1\ifx#2.#2\else\ifx#2,#2\else\/#2\fi\fi}}
\long\def\monoespaco#1{{\fam7\tentt@ #1}}

% Comandos de justificação e texto:
\def\alinhaesquerda{\rightskip 0em plus 2em \leftskip 0pt \spaceskip.3333em%
     \xspaceskip.5em\relax}
\def\alinhadireita{\leftskip 0em plus 2em \rightskip0pt \spaceskip.3333em%
     \xspaceskip.5em\relax}
\def\alinhacentro{\rightskip 0em plus 2em \leftskip 0em plus 2em%
                  \spaceskip.3333em \xspaceskip.5em\relax}
\def\alinhanormal{\leftskip0pt\rightskip0pt\spaceskip0pt\xspaceskip0pt\relax}

% Magitex não interage com usuário via prompt em caso de erro:
\nonstopmode

% Conversão para caracteres maiúsculos e minúsculos:
\def\maiusculo{\uppercase}
\def\minusculo{\lowercase}

% Comandos de datas:
\def\ano{\the\year}
\def\mes{\the\month}
\def\dia{\the\day}

% Inserção de novos arquivos:
\def\insere#1{\input #1}

% O número da página:
\def\pagina{\the\pageno@}

% Gera uma linha horizontal:
\def\linha{{\hrule}}

%%%%%%%%%%%%%%%%%%%%%%%%%%%%%%%%%%%%%%%%%%%%%%%%%%%%%%%%%%%%%%%%%%%%%%%%%%
%%%%%%%%%%%%%%%%%%%%%%%%%%%%%%%%%%%%%%%%%%%%%%%%%%%%%%%%%%%%%%%%%%%%%%%%%%
% O código abaixo é do Plain TeX e está aqui como referência. Ele será %%%
% substituído por código novo (ou não) à medida que o MAgiTeX se       %%%
% desenvolver                                                          %%%
%%%%%%%%%%%%%%%%%%%%%%%%%%%%%%%%%%%%%%%%%%%%%%%%%%%%%%%%%%%%%%%%%%%%%%%%%%
%%%%%%%%%%%%%%%%%%%%%%%%%%%%%%%%%%%%%%%%%%%%%%%%%%%%%%%%%%%%%%%%%%%%%%%%%%


% To make the plain macros more efficient in time and space,
% several constant values are declared here as control sequences.
% If they were changed, anything could happen; so they are private symbols.
\chardef\sixt@@n=16
\chardef\@cclv=255
\mathchardef\@cclvi=256
\mathchardef\@m=1000
\mathchardef\@M=10000
\mathchardef\@MM=20000

% Allocation of registers

\count15=9 % allocates \toks registers 10, 11, ...
\count16=-1 % allocates input streams 0, 1, ...
\count17=-1 % allocates output streams 0, 1, ...
\count18=3 % allocates math families 4, 5, ...
\count19=0 % allocates \language codes 1, 2, ...
\count20=255 % allocates insertions 254, 253, ...
\countdef\insc@unt=20 % the insertion counter
\countdef\allocationnumber=21 % the most recent allocation
\countdef\m@ne=22 \m@ne=-1 % a handy constant
\def\wlog{\immediate\write\m@ne} % write on log file (only)

% Here are abbreviations for the names of scratch registers
% that don't need to be allocated.



% Now, we define \newcount, \newbox, etc. so that you can say \newcount\foo
% and \foo will be defined (with \countdef) to be the next counter.
% To find out which counter \foo is, you can look at \allocationnumber.
% Since there's no \boxdef command, \chardef is used to define a \newbox,
% \newinsert, \newfam, and so on.
\outer\def\newfam{\alloc@8\fam\chardef\sixt@@n}
\outer\def\newlanguage{\alloc@9\language\chardef\@cclvi}
\def\alloc@#1#2#3#4#5{\global\advance\count1#1by1
  \ch@ck#1#4#2% make sure there's still room
  \allocationnumber=\count1#1%
  \global#3#5=\allocationnumber
  \wlog{\string#5=\string#2\the\allocationnumber}}
\outer\def\newinsert#1{\global\advance\insc@unt by\m@ne
  \ch@ck0\insc@unt\count
  \ch@ck1\insc@unt\dimen
  \ch@ck2\insc@unt\skip
  \ch@ck4\insc@unt\box
  \allocationnumber=\insc@unt
  \global\chardef#1=\allocationnumber
  \wlog{\string#1=\string\insert\the\allocationnumber}}
\def\ch@ck#1#2#3{\ifnum\count1#1<#2%
  \else\errmessage{No room for a new #3}\fi}

% Here are some examples of allocation.


% And here's a different sort of allocation:
% For example, \newif\iffoo creates \footrue, \foofalse to go with \iffoo.
\outer\def\newif#1{\count@\escapechar \escapechar\m@ne
  \expandafter\expandafter\expandafter
   \def\@if#1{true}{\let#1=\iftrue}%
  \expandafter\expandafter\expandafter
   \def\@if#1{false}{\let#1=\iffalse}%
  \@if#1{false}\escapechar\count@} % the condition starts out false
\def\@if#1#2{\csname\expandafter\if@\string#1#2\endcsname}
{\uccode`1=`i \uccode`2=`f \uppercase{\gdef\if@12{}}} % `if' is required

% Assign initial values to TeX's parameters

\message{parameters,}

% All of TeX's numeric parameters are listed here,
% but the code is commented out if no special value needs to be set.
% INITEX makes all parameters zero except where noted.

\pretolerance=100
\tolerance=200 % INITEX sets this to 10000
\hbadness=1000
\vbadness=1000
\linepenalty=10
\hyphenpenalty=50
\exhyphenpenalty=50
\binoppenalty=700
\relpenalty=500
\clubpenalty=150
\widowpenalty=150
\displaywidowpenalty=50
\brokenpenalty=100
\predisplaypenalty=10000
% \postdisplaypenalty=0
% \interlinepenalty=0
% \floatingpenalty=0, set during \insert
% \outputpenalty=0, set before TeX enters \output
\doublehyphendemerits=10000
\finalhyphendemerits=5000
\adjdemerits=10000
% \looseness=0, cleared by TeX after each paragraph
% \pausing=0
% \holdinginserts=0
% \tracingonline=0
% \tracingmacros=0
% \tracingstats=0
% \tracingparagraphs=0
% \tracingpages=0
% \tracingoutput=0
\tracinglostchars=1
% \tracingcommands=0
% \tracingrestores=0
% \language=0
\uchyph=1
% \lefthyphenmin=2 \righthyphenmin=3 set below
% \globaldefs=0
% \maxdeadcycles=25 % INITEX does this
% \hangafter=1 % INITEX does this, also TeX after each paragraph
% \fam=0
% \mag=1000 % INITEX does this
% \escapechar=`\\ % INITEX does this
\defaultskewchar=-1
% \endlinechar=`\^^M % INITEX does this
\newlinechar=-1
\delimiterfactor=901
% \time=now % TeX does this at beginning of job
% \day=now % TeX does this at beginning of job
% \month=now % TeX does this at beginning of job
% \year=now % TeX does this at beginning of job
\showboxbreadth=5
\showboxdepth=3
\errorcontextlines=5

\hfuzz=0.1pt
\vfuzz=0.1pt
\overfullrule=5pt
\hsize=6.5in
\vsize=8.9in
\maxdepth=4pt
\splitmaxdepth=\maxdimen
\boxmaxdepth=\maxdimen
% \lineskiplimit=0pt, changed by \normalbaselines
\delimitershortfall=5pt
\nulldelimiterspace=1.2pt
\scriptspace=0.5pt
% \mathsurround=0pt
% \predisplaysize=0pt, set before TeX enters $$
% \displaywidth=0pt, set before TeX enters $$
% \displayindent=0pt, set before TeX enters $$
\parindent=20pt
% \hangindent=0pt, zeroed by TeX after each paragraph
% \hoffset=0pt
% \voffset=0pt

% \baselineskip=0pt, changed by \normalbaselines
% \lineskip=0pt, changed by \normalbaselines
\parskip=0pt plus 1pt
\abovedisplayskip=12pt plus 3pt minus 9pt
\abovedisplayshortskip=0pt plus 3pt
\belowdisplayskip=12pt plus 3pt minus 9pt
\belowdisplayshortskip=7pt plus 3pt minus 4pt
% \leftskip=0pt
% \rightskip=0pt
\topskip=10pt
\splittopskip=10pt
% \tabskip=0pt
% \spaceskip=0pt
% \xspaceskip=0pt
\parfillskip=0pt plus 1fil

\thinmuskip=3mu
\medmuskip=4mu plus 2mu minus 4mu
\thickmuskip=5mu plus 5mu

% We also define special registers that function like parameters:

% Definitions for preloaded fonts

\def\magstephalf{1095 }
\def\magstep#1{\ifcase#1 \@m\or 1200\or 1440\or 1728\or 2074\or 2488\fi\relax}

% Fonts assigned to \preloaded are not part of "plain TeX",
% but they are preloaded so that other format packages can use them.
% For example, if another set of macros says "\font\ninerm=cmr9",
% TeX will not have to reload the font metric information for cmr9.


\message{more fonts,}
% Additional \preloaded fonts can be specified here.
% (And those that were \preloaded above can be eliminated.)

\skewchar\teni@='177 \skewchar\seveni@='177 \skewchar\fivei@='177
\skewchar\tensy@='60 \skewchar\sevensy@='60 \skewchar\fivesy@='60

\textfont0=\tenrm@ \scriptfont0=\sevenrm@ \scriptscriptfont0=\fiverm@
\textfont1=\teni@ \scriptfont1=\seveni@ \scriptscriptfont1=\fivei@
\def\mit{\fam1} \def\oldstyle{\fam1\teni@}
\textfont2=\tensy@ \scriptfont2=\sevensy@ \scriptscriptfont2=\fivesy@
\def\cal{\fam2}
\textfont3=\tenex@ \scriptfont3=\tenex@ \scriptscriptfont3=\tenex@
\newfam\itfam % \it is family 4
\textfont\itfam=\tenit@
\newfam\slfam % \sl is family 5
\textfont\slfam=\tensl@
\newfam\bffam  % \bf is family 6
\textfont\bffam=\tenbf@ \scriptfont\bffam=\sevenbf@
\scriptscriptfont\bffam=\fivebf@
\newfam\ttfam  % \tt is family 7
\textfont\ttfam=\tentt@



% Macros for setting ordinary text
\message{macros,}

\def\normalbaselines{\lineskip\normallineskip
  \baselineskip\normalbaselineskip \lineskiplimit\normallineskiplimit}

\def\^^M{\ } % control <return> = control <space>
\def\^^I{\ } % same for <tab>

\def\lbrack{[} \def\rbrack{]}

\let\endgraf=\par \let\endline=\cr

\def\space{ }
\def\empty{}
\def\null{\hbox{}}

\let\bgroup={ \let\egroup=}

% In \obeylines, we say `\let^^M=\par' instead of `\def^^M{\par}'
% since this allows, for example, `\let\par=\cr \obeylines \halign{...'
{\catcode`\^^M=13 % these lines must end with %
  \gdef\obeylines{\catcode`\^^M13 \let^^M\par}%
  \global\let^^M\par} % this is in case ^^M appears in a \write
\def\obeyspaces{\catcode`\ 13}
{\obeyspaces\global\let =\space}

\def\loop#1\repeat{\def\body{#1}\iterate}
\def\iterate{\body \let\next\iterate \else\let\next\relax\fi \next}
\let\repeat=\fi % this makes \loop...\if...\repeat skippable

\def\negthinspace{\kern-.16667em }
\def\enspace{\kern.5em }

\def\enskip{\hskip.5em\relax}
\def\quad{\hskip1em\relax}
\def\qquad{\hskip2em\relax}


\def\nointerlineskip{\prevdepth-1000\p@}
\def\offinterlineskip{\baselineskip-1000\p@
  \lineskip\z@ \lineskiplimit\maxdimen}

\def\topglue{\nointerlineskip\vglue-\topskip\vglue} % for top of page
\def\vglue{\afterassignment\vgl@\skip@=}
\def\vgl@{\par \dimen@\prevdepth \hrule height\z@
  \nobreak\vskip\skip@ \prevdepth\dimen@}
\def\hglue{\afterassignment\hgl@\skip@=}
\def\hgl@{\leavevmode \count@\spacefactor \vrule width\z@
  \nobreak\hskip\skip@ \spacefactor\count@}

\def~{\penalty\@M \ } % tie
\def\slash{/\penalty\exhyphenpenalty} % a `/' that acts like a `-'

\def\break{\penalty-\@M}
\def\nobreak{\penalty \@M}
\def\allowbreak{\penalty \z@}

\def\filbreak{\par\vfil\penalty-200\vfilneg}
\def\goodbreak{\par\penalty-500 }
\def\eject{\par\break}
\def\supereject{\par\penalty-\@MM}

\def\removelastskip{\ifdim\lastskip=\z@\else\vskip-\lastskip\fi}
\def\smallbreak{\par\ifdim\lastskip<\smallskipamount
  \removelastskip\penalty-50\smallskip\fi}
\def\medbreak{\par\ifdim\lastskip<\medskipamount
  \removelastskip\penalty-100\medskip\fi}
\def\bigbreak{\par\ifdim\lastskip<\bigskipamount
  \removelastskip\penalty-200\bigskip\fi}

\def\rlap#1{\hbox to\z@{#1\hss}}
\def\llap#1{\hbox to\z@{\hss#1}}

\def\m@th{\mathsurround\z@}
\def\underbar#1{$\setbox\z@\hbox{#1}\dp\z@\z@
  \m@th \underline{\box\z@}$}


\def\strut{\relax\ifmmode\copy\strutbox\else\unhcopy\strutbox\fi}

\def\hidewidth{\hskip\hideskip} % for alignment entries that can stick out
\def\ialign{\everycr{}\tabskip\z@skip\halign} % initialized \halign
\def\multispan#1{\omit \mscount#1\relax
  \loop\ifnum\mscount>1 \sp@n\repeat}
\def\sp@n{\span\omit\advance\mscount\m@ne}

\newif\ifus@ \newif\if@cr


\def\cleartabs{\global\setbox\tabsyet\null \setbox\tabs\null}
\def\settabs{\setbox\tabs\null \futurelet\next\sett@b}
\let\+=\relax % in case this file is being read in twice
\def\sett@b{\ifx\next\+\def\nxt{\afterassignment\s@tt@b\let\nxt}%
  \else\let\nxt\s@tcols\fi \let\next\relax \nxt}
\def\s@tt@b{\let\nxt\relax \us@false\m@ketabbox}
\def\tabalign{\us@true\m@ketabbox} % non-\outer version of \+
\outer\def\+{\tabalign}
\def\s@tcols#1\columns{\count@#1\dimen@\hsize
  \loop\ifnum\count@>\z@ \@nother \repeat}
\def\@nother{\dimen@ii\dimen@ \divide\dimen@ii\count@
  \setbox\tabs\hbox{\hbox to\dimen@ii{}\unhbox\tabs}%
  \advance\dimen@-\dimen@ii \advance\count@\m@ne}

\def\m@ketabbox{\begingroup
  \global\setbox\tabsyet\copy\tabs
  \global\setbox\tabsdone\null
  \def\cr{\@crtrue\crcr\egroup\egroup
    \ifus@\unvbox\z@\lastbox\fi\endgroup
    \setbox\tabs\hbox{\unhbox\tabsyet\unhbox\tabsdone}}%
  \setbox\z@\vbox\bgroup\@crfalse
    \ialign\bgroup&\t@bbox##\t@bb@x\crcr}

\def\t@bbox{\setbox\z@\hbox\bgroup}
\def\t@bb@x{\if@cr\egroup % now \box\z@ holds the column
  \else\hss\egroup \global\setbox\tabsyet\hbox{\unhbox\tabsyet
      \global\setbox1\lastbox}% now \box1 holds its size
    \ifvoid1\global\setbox1\hbox to\wd\z@{}%
    \else\setbox\z@\hbox to\wd1{\unhbox\z@}\fi
    \global\setbox\tabsdone\hbox{\box1\unhbox\tabsdone}\fi
  \box\z@}

\def\hang{\hangindent\parindent}
\def\textindent#1{\indent\llap{#1\enspace}\ignorespaces}
\def\item{\par\hang\textindent}
\def\itemitem{\par\indent \hangindent2\parindent \textindent}
\def\narrower{\advance\leftskip\parindent
  \advance\rightskip\parindent}

\outer\def\beginsection#1\par{\vskip\z@ plus.3\vsize\penalty-250
  \vskip\z@ plus-.3\vsize\bigskip\vskip\parskip
  \message{#1}\line{{\fam6\tenbf@#1}\hss}\nobreak\smallskip\noindent}
\outer\def\proclaim #1. #2\par{\medbreak
  \noindent{\fam6\tenbf@#1.\enspace}{\fam5\tensl@#2\par}%
  \ifdim\lastskip<\medskipamount \removelastskip\penalty55\medskip\fi}

\def\raggedright{\rightskip\z@ plus2em \spaceskip.3333em \xspaceskip.5em\relax}
\def\ttraggedright{\fam7\tentt@\rightskip\z@ plus2em\relax} % for use with \tt only

\chardef\%=`\%
\chardef\&=`\&
\chardef\#=`\#
\chardef\$=`\$
\chardef\ss="19
\chardef\oe="1B
\chardef\OE="1E
\chardef\i="10 \chardef\j="11 % dotless letters
\def\aa{\accent23a}
\def\l{\char32l}
\def\L{\leavevmode\setbox0\hbox{L}\hbox to\wd0{\hss\char32L}}

\def\leavevmode{\unhbox\voidb@x} % begins a paragraph, if necessary
\def\_{\leavevmode \kern.06em \vbox{\hrule width.3em}}
\def\AA{\leavevmode\setbox0\hbox{!}\dimen@\ht0\advance\dimen@-1ex%
  \rlap{\raise.67\dimen@\hbox{\char'27}}A}

\def\mathhexbox#1#2#3{\leavevmode
  \hbox{$\m@th \mathchar"#1#2#3$}}
\def\dag{\mathhexbox279}
\def\ddag{\mathhexbox27A}
\def\S{\mathhexbox278}
\def\P{\mathhexbox27B}
\def\Orb{\mathhexbox20D}

\def\oalign#1{\leavevmode\vtop{\baselineskip\z@skip \lineskip.25ex%
  \ialign{##\crcr#1\crcr}}} \def\o@lign{\lineskiplimit\z@ \oalign}
\def\ooalign{\lineskiplimit-\maxdimen \oalign} % chars over each other
{\catcode`p=12 \catcode`t=12 \gdef\\#1pt{#1}} \let\getf@ctor=\\
\def\sh@ft#1{\dimen@#1\kern\expandafter\getf@ctor\the\fontdimen1\font
  \dimen@} % kern by #1 times the current slant
\def\d#1{{\o@lign{\relax#1\crcr\hidewidth\sh@ft{-1ex}.\hidewidth}}}
\def\b#1{{\o@lign{\relax#1\crcr\hidewidth\sh@ft{-3ex}%
    \vbox to.2ex{\hbox{\char22}\vss}\hidewidth}}}
\def\c#1{{\setbox\z@\hbox{#1}\ifdim\ht\z@=1ex\accent24 #1%
  \else\ooalign{\unhbox\z@\crcr\hidewidth\char24\hidewidth}\fi}}
\def\copyright{{\ooalign{\hfil\raise.07ex\hbox{c}\hfil\crcr\Orb}}}

\def\dots{\relax\ifmmode\ldots\else$\m@th\ldots\,$\fi}

\def\v#1{{\accent20 #1}} \let\^^_=\v
\def\u#1{{\accent21 #1}} \let\^^S=\u
\def\=#1{{\accent22 #1}}
\def\.#1{{\accent95 #1}}
\def\H#1{{\accent"7D #1}}
\def\t#1{{\edef\next{\the\font}\the\textfont1\accent"7F\next#1}}

\def\hrulefill{\leaders\hrule\hfill}
\def\dotfill{\cleaders\hbox{$\m@th \mkern1.5mu.\mkern1.5mu$}\hfill}
\def\rightarrowfill{$\m@th\smash-\mkern-7mu%
  \cleaders\hbox{$\mkern-2mu\smash-\mkern-2mu$}\hfill
  \mkern-7mu\mathord\rightarrow$}
\def\leftarrowfill{$\m@th\mathord\leftarrow\mkern-7mu%
  \cleaders\hbox{$\mkern-2mu\smash-\mkern-2mu$}\hfill
  \mkern-7mu\smash-$}
\mathchardef\braceld="37A \mathchardef\bracerd="37B
\mathchardef\bracelu="37C \mathchardef\braceru="37D
\def\downbracefill{$\m@th \setbox\z@\hbox{$\braceld$}%
  \braceld\leaders\vrule height\ht\z@ depth\z@\hfill\braceru
  \bracelu\leaders\vrule height\ht\z@ depth\z@\hfill\bracerd$}
\def\upbracefill{$\m@th \setbox\z@\hbox{$\braceld$}%
  \bracelu\leaders\vrule height\ht\z@ depth\z@\hfill\bracerd
  \braceld\leaders\vrule height\ht\z@ depth\z@\hfill\braceru$}

% Macros for math setting
\message{math definitions,}

\let\sp=^ \let\sb=_
\def\,{\mskip\thinmuskip}
\def\>{\mskip\medmuskip}
\def\;{\mskip\thickmuskip}
\def\!{\mskip-\thinmuskip}
\def\*{\discretionary{\espacinho\the\textfont2\char2}{}{}}
{\catcode`\'=13 \gdef'{^\bgroup\prim@s}}
\def\prim@s{\prime\futurelet\next\pr@m@s}
\def\pr@m@s{\ifx'\next\let\nxt\pr@@@s \else\ifx^\next\let\nxt\pr@@@t
  \else\let\nxt\egroup\fi\fi \nxt}
\def\pr@@@s#1{\prim@s} \def\pr@@@t#1#2{#2\egroup}
{\catcode`\^^Z=13 \gdef^^Z{\not=}} % ^^Z is like \ne in math

{\catcode`\_=13 \global\let_=\_} % _ in math is either subscript or \_

\mathchardef\alpha="010B
\mathchardef\beta="010C
\mathchardef\gamma="010D
\mathchardef\delta="010E
\mathchardef\epsilon="010F
\mathchardef\zeta="0110
\mathchardef\eta="0111
\mathchardef\theta="0112
\mathchardef\iota="0113
\mathchardef\kappa="0114
\mathchardef\lambda="0115
\mathchardef\mu="0116
\mathchardef\nu="0117
\mathchardef\xi="0118
\mathchardef\pi="0119
\mathchardef\rho="011A
\mathchardef\sigma="011B
\mathchardef\tau="011C
\mathchardef\upsilon="011D
\mathchardef\phi="011E
\mathchardef\chi="011F
\mathchardef\psi="0120
\mathchardef\omega="0121
\mathchardef\varepsilon="0122
\mathchardef\vartheta="0123
\mathchardef\varpi="0124
\mathchardef\varrho="0125
\mathchardef\varsigma="0126
\mathchardef\varphi="0127
\mathchardef\Gamma="7000
\mathchardef\Delta="7001
\mathchardef\Theta="7002
\mathchardef\Lambda="7003
\mathchardef\Xi="7004
\mathchardef\Pi="7005
\mathchardef\Sigma="7006
\mathchardef\Upsilon="7007
\mathchardef\Phi="7008
\mathchardef\Psi="7009
\mathchardef\Omega="700A

\mathchardef\aleph="0240
\def\hbar{{\mathchar'26\mkern-9muh}}
\mathchardef\imath="017B
\mathchardef\jmath="017C
\mathchardef\ell="0160
\mathchardef\wp="017D
\mathchardef\Re="023C
\mathchardef\Im="023D
\mathchardef\partial="0140
\mathchardef\infty="0231
\mathchardef\prime="0230
\mathchardef\emptyset="023B
\mathchardef\nabla="0272
\def\surd{{\mathchar"1270}}
\mathchardef\top="023E
\mathchardef\bot="023F
\def\angle{{\vbox{\ialign{$\m@th\scriptstyle##$\crcr
      \not\mathrel{\mkern14mu}\crcr
      \noalign{\nointerlineskip}
      \mkern2.5mu\leaders\hrule height.34pt\hfill\mkern2.5mu\crcr}}}}
\mathchardef\triangle="0234
\mathchardef\forall="0238
\mathchardef\exists="0239
\mathchardef\neg="023A \let\lnot=\neg
\mathchardef\flat="015B
\mathchardef\natural="015C
\mathchardef\sharp="015D
\mathchardef\clubsuit="027C
\mathchardef\diamondsuit="027D
\mathchardef\heartsuit="027E
\mathchardef\spadesuit="027F

\mathchardef\coprod="1360
\mathchardef\bigvee="1357
\mathchardef\bigwedge="1356
\mathchardef\biguplus="1355
\mathchardef\bigcap="1354
\mathchardef\bigcup="1353
\mathchardef\intop="1352 \def\int{\intop\nolimits}
\mathchardef\prod="1351
\mathchardef\sum="1350
\mathchardef\bigotimes="134E
\mathchardef\bigoplus="134C
\mathchardef\bigodot="134A
\mathchardef\ointop="1348 \def\oint{\ointop\nolimits}
\mathchardef\bigsqcup="1346
\mathchardef\smallint="1273

\mathchardef\triangleleft="212F
\mathchardef\triangleright="212E
\mathchardef\bigtriangleup="2234
\mathchardef\bigtriangledown="2235
\mathchardef\wedge="225E \let\land=\wedge
\mathchardef\vee="225F \let\lor=\vee
\mathchardef\cap="225C
\mathchardef\cup="225B
\mathchardef\ddagger="227A
\mathchardef\dagger="2279
\mathchardef\sqcap="2275
\mathchardef\sqcup="2274
\mathchardef\uplus="225D
\mathchardef\amalg="2271
\mathchardef\diamond="2205
\mathchardef\bullet="220F
\mathchardef\wr="226F
\mathchardef\div="2204
\mathchardef\odot="220C
\mathchardef\oslash="220B
\mathchardef\otimes="220A
\mathchardef\ominus="2209
\mathchardef\oplus="2208
\mathchardef\mp="2207
\mathchardef\pm="2206
\mathchardef\circ="220E
\mathchardef\bigcirc="220D
\mathchardef\setminus="226E % for set difference A\setminus B
\mathchardef\cdot="2201
\mathchardef\ast="2203
\mathchardef\times="2202
\mathchardef\star="213F

\mathchardef\propto="322F
\mathchardef\sqsubseteq="3276
\mathchardef\sqsupseteq="3277
\mathchardef\parallel="326B
\mathchardef\mid="326A
\mathchardef\dashv="3261
\mathchardef\vdash="3260
\mathchardef\nearrow="3225
\mathchardef\searrow="3226
\mathchardef\nwarrow="322D
\mathchardef\swarrow="322E
\mathchardef\Leftrightarrow="322C
\mathchardef\Leftarrow="3228
\mathchardef\Rightarrow="3229
\def\neq{\not=} \let\ne=\neq
\mathchardef\leq="3214 \let\le=\leq
\mathchardef\geq="3215 \let\ge=\geq
\mathchardef\succ="321F
\mathchardef\prec="321E
\mathchardef\approx="3219
\mathchardef\succeq="3217
\mathchardef\preceq="3216
\mathchardef\supset="321B
\mathchardef\subset="321A
\mathchardef\supseteq="3213
\mathchardef\subseteq="3212
\mathchardef\in="3232
\mathchardef\ni="3233 \let\owns=\ni
\mathchardef\gg="321D
\mathchardef\ll="321C
\mathchardef\not="3236
\mathchardef\leftrightarrow="3224
\mathchardef\leftarrow="3220 \let\gets=\leftarrow
\mathchardef\rightarrow="3221 \let\to=\rightarrow
\mathchardef\mapstochar="3237 \def\mapsto{\mapstochar\rightarrow}
\mathchardef\sim="3218
\mathchardef\simeq="3227
\mathchardef\perp="323F
\mathchardef\equiv="3211
\mathchardef\asymp="3210
\mathchardef\smile="315E
\mathchardef\frown="315F
\mathchardef\leftharpoonup="3128
\mathchardef\leftharpoondown="3129
\mathchardef\rightharpoonup="312A
\mathchardef\rightharpoondown="312B

\def\joinrel{\mathrel{\mkern-3mu}}
\def\relbar{\mathrel{\smash-}} % \smash, because - has the same height as +
\def\Relbar{\mathrel=}
\mathchardef\lhook="312C \def\hookrightarrow{\lhook\joinrel\rightarrow}
\mathchardef\rhook="312D \def\hookleftarrow{\leftarrow\joinrel\rhook}
\def\bowtie{\mathrel\triangleright\joinrel\mathrel\triangleleft}
\def\models{\mathrel|\joinrel=}
\def\Longrightarrow{\Relbar\joinrel\Rightarrow}
\def\longrightarrow{\relbar\joinrel\rightarrow}
\def\longleftarrow{\leftarrow\joinrel\relbar}
\def\Longleftarrow{\Leftarrow\joinrel\Relbar}
\def\longmapsto{\mapstochar\longrightarrow}
\def\longleftrightarrow{\leftarrow\joinrel\rightarrow}
\def\Longleftrightarrow{\Leftarrow\joinrel\Rightarrow}
\def\iff{\;\Longleftrightarrow\;}

\mathchardef\ldotp="613A % ldot as a punctuation mark
\mathchardef\cdotp="6201 % cdot as a punctuation mark
\mathchardef\colon="603A % colon as a punctuation mark
\def\ldots{\mathinner{\ldotp\ldotp\ldotp}}
\def\cdots{\mathinner{\cdotp\cdotp\cdotp}}
\def\vdots{\vbox{\baselineskip4\p@ \lineskiplimit\z@
    \kern6\p@\hbox{.}\hbox{.}\hbox{.}}}
\def\ddots{\mathinner{\mkern1mu\raise7\p@\vbox{\kern7\p@\hbox{.}}\mkern2mu
    \raise4\p@\hbox{.}\mkern2mu\raise\p@\hbox{.}\mkern1mu}}

\def\acute{\mathaccent"7013 }
\def\grave{\mathaccent"7012 }
\def\ddot{\mathaccent"707F }
\def\tilde{\mathaccent"707E }
\def\bar{\mathaccent"7016 }
\def\breve{\mathaccent"7015 }
\def\check{\mathaccent"7014 }
\def\hat{\mathaccent"705E }
\def\vec{\mathaccent"017E }
\def\dot{\mathaccent"705F }
\def\widetilde{\mathaccent"0365 }
\def\widehat{\mathaccent"0362 }
\def\overrightarrow#1{\vbox{\m@th\ialign{##\crcr
      \rightarrowfill\crcr\noalign{\kern-\p@\nointerlineskip}
      $\hfil\displaystyle{#1}\hfil$\crcr}}}
\def\overleftarrow#1{\vbox{\m@th\ialign{##\crcr
      \leftarrowfill\crcr\noalign{\kern-\p@\nointerlineskip}
      $\hfil\displaystyle{#1}\hfil$\crcr}}}
\def\overbrace#1{\mathop{\vbox{\m@th\ialign{##\crcr\noalign{\kern3\p@}
      \downbracefill\crcr\noalign{\kern3\p@\nointerlineskip}
      $\hfil\displaystyle{#1}\hfil$\crcr}}}\limits}
\def\underbrace#1{\mathop{\vtop{\m@th\ialign{##\crcr
      $\hfil\displaystyle{#1}\hfil$\crcr\noalign{\kern3\p@\nointerlineskip}
      \upbracefill\crcr\noalign{\kern3\p@}}}}\limits}
\def\skew#1#2#3{{\muskip\z@#1mu\divide\muskip\z@2 \mkern\muskip\z@
    #2{\mkern-\muskip\z@{#3}\mkern\muskip\z@}\mkern-\muskip\z@}{}}

\def\lmoustache{\delimiter"437A340 } % top from (, bottom from )
\def\rmoustache{\delimiter"537B341 } % top from ), bottom from (
\def\lgroup{\delimiter"462833A } % extensible ( with sharper tips
\def\rgroup{\delimiter"562933B } % extensible ) with sharper tips
\def\arrowvert{\delimiter"26A33C } % arrow without arrowheads
\def\Arrowvert{\delimiter"26B33D } % double arrow without arrowheads
\def\bracevert{\delimiter"77C33E } % the vertical bar that extends braces
\def\Vert{\delimiter"26B30D } \let\|=\Vert
\def\vert{\delimiter"26A30C }
\def\uparrow{\delimiter"3222378 }
\def\downarrow{\delimiter"3223379 }
\def\updownarrow{\delimiter"326C33F }
\def\Uparrow{\delimiter"322A37E }
\def\Downarrow{\delimiter"322B37F }
\def\Updownarrow{\delimiter"326D377 }
\def\backslash{\delimiter"26E30F } % for double coset G\backslash H
\def\rangle{\delimiter"526930B }
\def\langle{\delimiter"426830A }
\def\rbrace{\delimiter"5267309 } \let\}=\rbrace
\def\lbrace{\delimiter"4266308 } \let\{=\lbrace
\def\rceil{\delimiter"5265307 }
\def\lceil{\delimiter"4264306 }
\def\rfloor{\delimiter"5263305 }
\def\lfloor{\delimiter"4262304 }



\def\bigl{\mathopen\big}
\def\bigm{\mathrel\big}
\def\bigr{\mathclose\big}
\def\Bigl{\mathopen\Big}
\def\Bigm{\mathrel\Big}
\def\Bigr{\mathclose\Big}
\def\biggl{\mathopen\bigg}
\def\biggm{\mathrel\bigg}
\def\biggr{\mathclose\bigg}
\def\Biggl{\mathopen\Bigg}
\def\Biggm{\mathrel\Bigg}
\def\Biggr{\mathclose\Bigg}
\def\big#1{{\hbox{$\left#1\vbox to8.5\p@{}\right.\n@space$}}}
\def\Big#1{{\hbox{$\left#1\vbox to11.5\p@{}\right.\n@space$}}}
\def\bigg#1{{\hbox{$\left#1\vbox to14.5\p@{}\right.\n@space$}}}
\def\Bigg#1{{\hbox{$\left#1\vbox to17.5\p@{}\right.\n@space$}}}
\def\n@space{\nulldelimiterspace\z@ \m@th}

\def\choose{\atopwithdelims()}
\def\brack{\atopwithdelims[]}
\def\brace{\atopwithdelims\{\}}

\def\sqrt{\radical"270370 }

\def\mathpalette#1#2{\mathchoice{#1\displaystyle{#2}}%
  {#1\textstyle{#2}}{#1\scriptstyle{#2}}{#1\scriptscriptstyle{#2}}}
\def\root#1\of{\setbox\rootbox
  \hbox{$\m@th\scriptscriptstyle{#1}$}\mathpalette\r@@t}
\def\r@@t#1#2{\setbox\z@\hbox{$\m@th#1\sqrt{#2}$}\dimen@\ht\z@
  \advance\dimen@-\dp\z@
  \mkern5mu\raise.6\dimen@\copy\rootbox \mkern-10mu\box\z@}
\newif\ifv@ \newif\ifh@
\def\vphantom{\v@true\h@false\ph@nt}
\def\hphantom{\v@false\h@true\ph@nt}
\def\phantom{\v@true\h@true\ph@nt}
\def\ph@nt{\ifmmode\def\next{\mathpalette\mathph@nt}%
  \else\let\next\makeph@nt\fi\next}
\def\makeph@nt#1{\setbox\z@\hbox{#1}\finph@nt}
\def\mathph@nt#1#2{\setbox\z@\hbox{$\m@th#1{#2}$}\finph@nt}
\def\finph@nt{\setbox2\null
  \ifv@ \ht2\ht\z@ \dp2\dp\z@\fi
  \ifh@ \wd2\wd\z@\fi \box2}
\def\mathstrut{\vphantom(}
\def\smash{\relax % \relax, in case this comes first in \halign
  \ifmmode\def\next{\mathpalette\mathsm@sh}\else\let\next\makesm@sh
  \fi\next}
\def\makesm@sh#1{\setbox\z@\hbox{#1}\finsm@sh}
\def\mathsm@sh#1#2{\setbox\z@\hbox{$\m@th#1{#2}$}\finsm@sh}
\def\finsm@sh{\ht\z@\z@ \dp\z@\z@ \box\z@}

\def\cong{\mathrel{\mathpalette\@vereq\sim}} % congruence sign
\def\@vereq#1#2{\lower.5\p@\vbox{\lineskiplimit\maxdimen\lineskip-.5\p@
    \ialign{$\m@th#1\hfil##\hfil$\crcr#2\crcr=\crcr}}}
\def\notin{\mathrel{\mathpalette\c@ncel\in}}
\def\c@ncel#1#2{\m@th\ooalign{$\hfil#1\mkern1mu/\hfil$\crcr$#1#2$}}
\def\rightleftharpoons{\mathrel{\mathpalette\rlh@{}}}
\def\rlh@#1{\vcenter{\m@th\hbox{\ooalign{\raise2pt
          \hbox{$#1\rightharpoonup$}\crcr
        $#1\leftharpoondown$}}}}
\def\buildrel#1\over#2{\mathrel{\mathop{\kern\z@#2}\limits^{#1}}}
\def\doteq{\buildrel\textstyle.\over=}

\def\log{\mathop{\fam0\tenrm@ log}\nolimits}
\def\lg{\mathop{\fam0\tenrm@ lg}\nolimits}
\def\ln{\mathop{\fam0\tenrm@ ln}\nolimits}
\def\lim{\mathop{\fam0\tenrm@ lim}}
\def\limsup{\mathop{\fam0\tenrm@ lim\,sup}}
\def\liminf{\mathop{\fam0\tenrm@ lim\,inf}}
\def\sin{\mathop{\fam0\tenrm@ sin}\nolimits}
\def\arcsin{\mathop{\fam0\tenrm@ arcsin}\nolimits}
\def\sinh{\mathop{\fam0\tenrm@ sinh}\nolimits}
\def\cos{\mathop{\fam0\tenrm@ cos}\nolimits}
\def\arccos{\mathop{\fam0\tenrm@ arccos}\nolimits}
\def\cosh{\mathop{\fam0\tenrm@ cosh}\nolimits}
\def\tan{\mathop{\fam0\tenrm@ tan}\nolimits}
\def\arctan{\mathop{\fam0\tenrm@ arctan}\nolimits}
\def\tanh{\mathop{\fam0\tenrm@ tanh}\nolimits}
\def\cot{\mathop{\fam0\tenrm@ cot}\nolimits}
\def\coth{\mathop{\fam0\tenrm@ coth}\nolimits}
\def\sec{\mathop{\fam0\tenrm@ sec}\nolimits}
\def\csc{\mathop{\fam0\tenrm@ csc}\nolimits}
\def\max{\mathop{\fam0\tenrm@ max}}
\def\min{\mathop{\fam0\tenrm@ min}}
\def\sup{\mathop{\fam0\tenrm@ sup}}
\def\inf{\mathop{\fam0\tenrm@ inf}}
\def\arg{\mathop{\fam0\tenrm@ arg}\nolimits}
\def\ker{\mathop{\fam0\tenrm@ ker}\nolimits}
\def\dim{\mathop{\fam0\tenrm@ dim}\nolimits}
\def\hom{\mathop{\fam0\tenrm@ hom}\nolimits}
\def\det{\mathop{\fam0\tenrm@ det}}
\def\exp{\mathop{\fam0\tenrm@ exp}\nolimits}
\def\Pr{\mathop{\fam0\tenrm@ Pr}}
\def\gcd{\mathop{\fam0\tenrm@ gcd}}
\def\deg{\mathop{\fam0\tenrm@ deg}\nolimits}

\def\bmod{\nonscript\mskip-\medmuskip\mkern5mu
  \mathbin{\fam0\tenrm@ mod}\penalty900\mkern5mu\nonscript\mskip-\medmuskip}
\def\pmod#1{\allowbreak\mkern18mu({\fam0\tenrm@ mod}\,\,#1)}



\def\cases#1{\left\{\,\vcenter{\normalbaselines\m@th
    \ialign{$##\hfil$&\quad##\hfil\crcr#1\crcr}}\right.}
\def\matrix#1{\null\,\vcenter{\normalbaselines\m@th
    \ialign{\hfil$##$\hfil&&\quad\hfil$##$\hfil\crcr
      \mathstrut\crcr\noalign{\kern-\baselineskip}
      #1\crcr\mathstrut\crcr\noalign{\kern-\baselineskip}}}\,}
\def\pmatrix#1{\left(\matrix{#1}\right)}
\setbox0=\hbox{\tenex@ B} \p@renwd=\wd0 % width of the big left (
\def\bordermatrix#1{\begingroup \m@th
  \setbox\z@\vbox{\def\cr{\crcr\noalign{\kern2\p@\global\let\cr\endline}}%
    \ialign{$##$\hfil\kern2\p@\kern\p@renwd&\espacinho\hfil$##$\hfil
      &&\quad\hfil$##$\hfil\crcr
      \omit\strut\hfil\crcr\noalign{\kern-\baselineskip}%
      #1\crcr\omit\strut\cr}}%
  \setbox2\vbox{\unvcopy\z@\global\setbox1\lastbox}%
  \setbox2\hbox{\unhbox1\unskip\global\setbox1\lastbox}%
  \setbox2\hbox{$\kern\wd1\kern-\p@renwd\left(\kern-\wd1
    \global\setbox1\vbox{\box1\kern2\p@}%
    \vcenter{\kern-\ht1\unvbox\z@\kern-\baselineskip}\,\right)$}%
  \null\;\vbox{\kern\ht1\box2}\endgroup}

\def\openup{\afterassignment\@penup\dimen@=}
\def\@penup{\advance\lineskip\dimen@
  \advance\baselineskip\dimen@
  \advance\lineskiplimit\dimen@}
\def\eqalign#1{\null\,\vcenter{\openup\jot\m@th
  \ialign{\strut\hfil$\displaystyle{##}$&$\displaystyle{{}##}$\hfil
      \crcr#1\crcr}}\,}
\newif\ifdt@p
\def\displ@y{\global\dt@ptrue\openup\jot\m@th
  \everycr{\noalign{\ifdt@p \global\dt@pfalse \ifdim\prevdepth>-1000\p@
      \vskip-\lineskiplimit \vskip\normallineskiplimit \fi
      \else \penalty\interdisplaylinepenalty \fi}}}
\def\@lign{\tabskip\z@skip\everycr{}} % restore inside \displ@y
\def\displaylines#1{\displ@y \tabskip\z@skip
  \halign{\hbox to\displaywidth{$\@lign\hfil\displaystyle##\hfil$}\crcr
    #1\crcr}}
\def\eqalignno#1{\displ@y \tabskip\centering
  \halign to\displaywidth{\hfil$\@lign\displaystyle{##}$\tabskip\z@skip
    &$\@lign\displaystyle{{}##}$\hfil\tabskip\centering
    &\llap{$\@lign##$}\tabskip\z@skip\crcr
    #1\crcr}}
\def\leqalignno#1{\displ@y \tabskip\centering
  \halign to\displaywidth{\hfil$\@lign\displaystyle{##}$\tabskip\z@skip
    &$\@lign\displaystyle{{}##}$\hfil\tabskip\centering
    &\kern-\displaywidth\rlap{$\@lign##$}\tabskip\displaywidth\crcr
    #1\crcr}}

% Definitions related to output

\message{output routines,}

\newif\ifr@ggedbottom
\def\raggedbottom{\topskip 10\p@ plus60\p@ \r@ggedbottomtrue}
\def\normalbottom{\topskip 10\p@ \r@ggedbottomfalse} % undoes \raggedbottom
\def\folio{\ifnum\pageno@<\z@ \romannumeral-\pageno@ \else\number\pageno@ \fi}
\def\nopagenumbers{\footline{\hfil}} % blank out the footline
\def\advancepageno{\ifnum\pageno@<\z@ \global\advance\pageno@\m@ne
  \else\global\advance\pageno@1 \fi} % increase |pageno|

\newinsert\footins
\def\footnote#1{\let\@sf\empty % parameter #2 (the text) is read later
  \ifhmode\edef\@sf{\spacefactor\the\spacefactor}\/\fi
  #1\@sf\vfootnote{#1}}
\def\vfootnote#1{\insert\footins\bgroup
  \interlinepenalty\interfootnotelinepenalty
  \splittopskip\ht\strutbox % top baseline for broken footnotes
  \splitmaxdepth\dp\strutbox \floatingpenalty\@MM
  \leftskip\z@skip \rightskip\z@skip \spaceskip\z@skip \xspaceskip\z@skip
  \textindent{#1}\footstrut\futurelet\next\fo@t}
\def\fo@t{\ifcat\bgroup\noexpand\next \let\next\f@@t
  \else\let\next\f@t\fi \next}
\def\f@@t{\bgroup\aftergroup\@foot\let\next}
\def\f@t#1{#1\@foot}
\def\@foot{\strut\egroup}
\def\footstrut{\vbox to\splittopskip{}}
\skip\footins=\bigskipamount % space added when footnote is present
\count\footins=1000 % footnote magnification factor (1 to 1)
\dimen\footins=8in % maximum footnotes per page

\newinsert\topins
\newif\ifp@ge \newif\if@mid
\def\topinsert{\@midfalse\p@gefalse\@ins}
\def\midinsert{\@midtrue\@ins}
\def\pageinsert{\@midfalse\p@getrue\@ins}
\skip\topins=\z@skip % no space added when a topinsert is present
\count\topins=1000 % magnification factor (1 to 1)
\dimen\topins=\maxdimen % no limit per page
\def\@ins{\par\begingroup\setbox\z@\vbox\bgroup} % start a \vbox
\def\endinsert{\egroup % finish the \vbox
  \if@mid \dimen@\ht\z@ \advance\dimen@\dp\z@ \advance\dimen@12\p@
    \advance\dimen@\pagetotal \advance\dimen@-\pageshrink
    \ifdim\dimen@>\pagegoal\@midfalse\p@gefalse\fi\fi
  \if@mid \bigskip\box\z@\bigbreak
  \else\insert\topins{\penalty100 % floating insertion
    \splittopskip\z@skip
    \splitmaxdepth\maxdimen \floatingpenalty\z@
    \ifp@ge \dimen@\dp\z@
    \vbox to\vsize{\unvbox\z@\kern-\dimen@}% depth is zero
    \else \box\z@\nobreak\bigskip\fi}\fi\endgroup}



\output{\plainoutput}
\def\plainoutput{\shipout\vbox{\makeheadline\pagebody\makefootline}%
  \advancepageno
  \ifnum\outputpenalty>-\@MM \else\dosupereject\fi}
\def\pagebody{\vbox to\vsize{\boxmaxdepth\maxdepth \pagecontents}}
\def\makeheadline{\vbox to\z@{\vskip-22.5\p@
  \hbox to\hsize{\vbox to8.5\p@{}\the\headline}\vss}\nointerlineskip}
\def\makefootline{\baselineskip24\p@\lineskiplimit\z@\hbox to\hsize{\the\footline}}
\def\dosupereject{\ifnum\insertpenalties>\z@ % something is being held over
  \hbox to\hsize{}\kern-\topskip\nobreak\vfill\supereject\fi}

\def\pagecontents{\ifvoid\topins\else\unvbox\topins\fi
  \dimen@=\dp\@cclv \unvbox\@cclv % open up \box255
  \ifvoid\footins\else % footnote info is present
    \vskip\skip\footins
    \footnoterule
    \unvbox\footins\fi
  \ifr@ggedbottom \kern-\dimen@ \vfil \fi}
\def\footnoterule{\kern-3\p@
  \hrule width 2truein \kern 2.6\p@} % the \hrule is .4pt high

% Hyphenation, miscellaneous macros, and initial values for standard layout
\message{hyphenation}

\lefthyphenmin=2 \righthyphenmin=3 % disallow x- or -xx breaks

\def\loggingall{\tracingcommands2\tracingstats2
  \tracingpages1\tracingoutput1\tracinglostchars1
  \tracingmacros2\tracingparagraphs1\tracingrestores1
  \showboxbreadth\maxdimen\showboxdepth\maxdimen}
\def\tracingall{\tracingonline1\loggingall}

\def\showhyphens#1{\setbox0\vbox{\parfillskip\z@skip\hsize\maxdimen\tenrm@
  \pretolerance\m@ne\tolerance\m@ne\hbadness0\showboxdepth0\ #1}}




\normalbaselines\fam0\tenrm@ % select roman font
\catcode`@=12 % at signs are no longer letters

\dump
